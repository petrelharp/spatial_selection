\documentclass{article}
\usepackage{fullpage}
\usepackage{amsmath,amssymb}

\begin{document}

\section{Introduction}

Recall questions of origin of adaptations

Summarize previous paper

Question assumptions of no standing variation and homogenous selection

Stick to constant-speed waves.

\section{Standing variation}

\subsection{Model}

selection pressure changes at $t=0$ from $1-s_d$ to $1+s_b$ -- homogeneous

rescale to two parameters 
-- a characteristic length of regions and a proportion of haplotypes (or of space?) that are from standing variation

Branching-process-sized point mass approx to local groups present at $t=0$ 
-- depends on local population density large enough relative to migration (to be "branching")
-- and migration not too large (since we approximate as a point mass).
Find mean density of clusters of successful standing mutants.
Is greater-than-one size of clusters likely to make a difference?

recall PPP computations; draw cones of opportunity for new mutations

\subsection{Results}

proportion of haplotypes (or of space?) that are from standing variation

mean (and variance?) of distance between haplotypes

influence of migration on things -- plot these versus migration


\section{Patchy selection}

\subsection{Model}

patches where selection is $1+s_b$ separated by $1-s_d$ -- separation large relative to migration.  
(or relative to $s_d/m$?)

Describe simple discrete model.

Approximation in continuous model by branching Brownian motion with killing to find migration rate between patches --
good approximation neglecting interaction (if local population size is large).
Look at Barton.

\subsection{Results}

Pull out results from other paper on discrete model.

when are patches effectively isolated?

\section{Applications}

Look at some human parameters?

\section{Discussion}

if populations and mutation rates are large enough, parallel adaptation is likely, aided by geography.
recall others' work.

patchy selection helps this happen;
effective migration rate looks like what;
strength of (positive?) selection suprisingly does not affect results (?)

standing variation will be important under these circumstances;
migration rate (and so geographic structure) will not affect numbers of types if this is true;
however migration always affcts spatial patterns.

discussion of applications.

\end{document}
