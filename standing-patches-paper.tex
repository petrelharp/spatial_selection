\documentclass{article}
\usepackage{fullpage}
\usepackage{amsmath,amssymb}

\begin{document}

\section{Introduction}

Recall questions of origin of adaptations

Reference on parallel adapation, standing variation, etc.:
 coop,
 pennings and hermisson,

Summarize previous paper

Question assumptions of no standing variation and homogenous selection

Stick to constant-speed waves.

\section{Standing variation}

\subsection{Model}

selection pressure changes at $t=0$ from $1-s_d$ to $1+s_b$ -- homogeneous

For now we ignore geographic hetergeneity, and assume that the species range
is a homogenous, two-dimensional region.  (results for a one-dimensional region also appear)
There are two selective types -- the neutral type, and the mutated type.
We assume that different mutations are distinguishable --
either as selectively equivalent mutations, or by linked variation.
At first, the mutated type is at a seletive disadvantage, for long enough to be at selection-mutation equilibrium,
but at a certain time the selective regime changes, so that the mutated type has a selective advantage and quickly spreads to fixation.
After fixation, alleles are either descended
from families of mutants present as standing variation when the selective regime changed,
or from new mutants arising since that time.

We will suppose that before time $t=0$,
the mutant type has fitness $1-s_d$ relative to the neutral type,
and that after time $t=0$,
the mutant type has fitness $1+s_b$,
for positive numbers $s_d$ and $s_b$.
We assume that diploid fitness is additive, or at lesat that the important dynamics are determined by the haploid fitness.
If the mutation is recessive, XXX more complicated things happen.
More concretely, suppose that the species occupies an area with mean density

rescale to two parameters
-- a characteristic length of regions and a proportion of haplotypes (or of space?) that are from standing variation

Branching-process-sized point mass approx to local groups present at $t=0$
-- depends on local population density large enough relative to migration (to be "branching")
-- and migration not too large (since we approximate as a point mass).
Find mean density of clusters of successful standing mutants. 
Is greater-than-one size of clusters likely to make a difference? 
 
recall PPP computations; draw cones of opportunity for new mutations 
 
\subsection{Results} 
 
proportion of haplotypes (or of space?) that are from standing variation 
 
mean (and variance?) of distance between haplotypes 
 
influence of migration on things -- plot these versus migration 
 
 
\section{Patchy selection} 
 
\subsection{Model} 
 
patches where selection is $1+s_b$ separated by $1-s_d$ -- separation large relative to migration.   
(or relative to $s_d/m$?) 
 
Describe simple discrete model. 
 
Approximation in continuous model by branching Brownian motion with killing to find migration rate between patches -- 
good approximation neglecting interaction (if local population size is large). 
Look at Barton. 
 
\subsection{Results} 
 
Pull out results from other paper on discrete model. 
 
when are patches effectively isolated? 
 
\section{Applications} 
 
Look at some human parameters? 
 
\section{Discussion} 
 
if populations and mutation rates are large enough, parallel adaptation is likely, aided by geography. 
recall others' work. 
 
patchy selection helps this happen; 
effective migration rate looks like what; 
strength of (positive?) selection suprisingly does not affect results (?) 
 
standing variation will be important under these circumstances; 
migration rate (and so geographic structure) will not affect numbers of types if this is true; 
however migration always affcts spatial patterns. 
 
discussion of applications. 
 
\end{document} 
