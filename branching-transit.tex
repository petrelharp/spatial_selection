\documentclass{article}
\usepackage{fullpage} 
\usepackage{graphicx}
\usepackage{amsmath}
\usepackage{amssymb}

\newcommand{\E}{\mathbb{E}}
\renewcommand{\P}{\mathbb{P}}
\newcommand{\var}{\mathop{\mbox{Var}}}
\newcommand{\Exp}{\mathop{\mbox{Exp}}}
\newcommand{\migrate}{\lambda_\text{mig}}

\begin{document}

Consider a largeish occupied patch of habitat in which an allele is beneficial ($s_b$),
surrounded by habitat where it is deleterious ($s_d$).
The allele is present in surrounding areas at low frequency due to migration-selection balance.
We want the first time that another patch at distance $x$ is successfully colonized.

The general picture:
\begin{itemize}

\item The excursions sufficiently far from the patch are like subcritical branching processes,
with mean $1-s_d$ and variance $\xi^2$.

\item An excursion that makes it to the new patch in $k$ copies establishes with probability $1-(1-p_s)^k$,
where $p_s \approx 2s_b/\xi^2$, so establishment has approximate probability $2 k s_b / \xi^2$.

\item The probability that the branching process lives for $t$ generations is asymptotically a constant ($\phi$, below) times $(1-s_d)^t$.

\item Conditioned on living for time $t$, the process can be represented by a single lineage whose offspring distribution
is the original distribution, conditioned to be nonzero, minus one.
(Also, it is known that conditional on being alive at $t\to\infty$, the allele is present in $K$ copies; where $K$ has mean $1/\phi$ and variance calculated below.)

\item Spatial motion is independent of branching;
so we can take the motion of the trunk lineage to be Brownian;
the chance that the motion first hits $x$ at time $t$ is (Borodin \& Salminen 2.02)
\begin{align}
\frac{x}{t^{3/2}\sqrt{2\pi}} \exp\left(-\frac{x^2}{2t}\right) .
\end{align}

\item Therefore, the probability that an excursion makes it to $x$ at time $t$ is asymptotically
\begin{align}
  (1-s_d)^t \phi \frac{x}{t^{3/2}\sqrt{2\pi}} \exp\left(-\frac{x^2}{2t}\right) ,
\end{align}
or the chance that it is still alive then time the chance the trunk lineage hits $x$ at that time.

\item This induces a distribution on $T$, the time that a successful excursion took to cross:
assuming, as is asymptotically correct, that the number of arriving lineages is independent of how long it took to get there,
if $\tau_x$ is the random time of arrival of a successful lineage,
\begin{align} \label{eqn:arrival_time_density}
 \P\{\tau_x \in dt\} = C_{s_d,x} \frac{(1-s_d)^t}{t^{3/2}} \exp\left(-\frac{x^2}{2t}\right) ,
\end{align}
where $C_{s_d,x}$ is a constant depending only on $s_d$ and $x$.


\item The length of the hitchhiking segment on \emph{one side} of the selected site $L$ is exponential with rate depending on $T$;
integrating over $T$ we get
(using Gradshteyn \& Ryzhik 3.471.9; note $\ell > 0 > \log(1-s_d)$):
\begin{align}
 \P\{ L > \ell \} &= \E\left[ e^{-\ell \tau_x} \right] \\
 & = C_{s_d,x} \int_0^\infty \frac{1}{t^{3/2}} \exp\left(-\frac{x^2}{2t} + t(\log(1-s_d) -\ell) \right)  dt \\
 &= C_{s_d,x} \frac{\sqrt{2\pi}}{x} e^{-x\sqrt{2(\ell-\log(1-s_d))}} ; \label{eqn:first_integral}
\end{align}
hence, defining $\alpha = - \log(1-s_d) \approx s_d$ (note $\alpha>0$),
\begin{align}
 \P\{ L > \ell \} &= \E\left[ e^{-\ell \tau_x} \right] = \exp\left\{{-x\left(\sqrt{2(\ell+\alpha)} - \sqrt{2\alpha}\right)}\right\} \qquad \mbox{for}\; \ell>0.
\end{align}
This is a shifted Weibull distribution, or equivalently, $L=X^2-\alpha$, where $X$ is a Exponential with rate $x\sqrt{2}$ conditioned to be at least $\sqrt{\alpha}$.
Note that the total length of a hitchhiking segment is the sum of two independent copies of this.

\item We could rescale with $x$ and get a limiting distribution for $\tau_x$:
\begin{align}
  \E\left[\exp\left( -\ell \frac{\tau_x}{x} \right)\right]
        &= \exp\left\{ - x\sqrt{2}\left(\sqrt{\frac{\ell}{x}+\alpha} - \sqrt{\alpha}\right) \right\} \\
        &\xrightarrow{x\to\infty} \exp\left\{ - \frac{\ell}{\sqrt{2\alpha}} \right\} ,
\end{align}
implying that $\tau_x/x \xrightarrow{a.s.} 1/\sqrt{2\alpha}$.

\item Ok, but how about the CLT?  Differentiating, we see that $\E[\tau_x]=x/\sqrt{2\alpha}$ and $\var[\tau_x]=x/2^{3/2}\alpha^{3/2}$,
which suggests looking at
\begin{align}
\E\left[\exp\left( -\ell \left(\frac{\tau_x-x/\sqrt{2\alpha}}{\sqrt{x}} \right) \right)\right]
    &= \exp\left\{ - x\sqrt{2}\left(\sqrt{\frac{\ell}{\sqrt{x}}+\alpha} - \sqrt{\alpha}\right) + \frac{\ell}{\sqrt{2\alpha x}} \right\} \\
    &\simeq \exp\left\{ - x\sqrt{2}\left( \frac{\ell}{2\sqrt{\alpha x}} - \frac{\ell^2}{8\alpha^{3/2}x}  \right) + \frac{\ell}{\sqrt{2\alpha x}} \right\} \\
    &= \exp\left\{ \frac{\ell^2}{2^{5/2}\alpha^{3/2}} \right\} ,
\end{align}
since $\sqrt{\alpha+\epsilon}=\sqrt{\alpha} +\epsilon/2\sqrt{\alpha}-\epsilon^2/8\alpha^{3/2} + O(\epsilon^3)$.
The Laplace transform of a $N(0,\sigma^2)$ is $\exp(\ell^2\sigma^2/2)$, so
\begin{align}
\frac{\tau_x - x/\sqrt{2\alpha}}{\sqrt{x}} \xrightarrow{d} N(0,2^{-3/2}\alpha^{-3/2}) \qquad \mbox{as}\;x\to\infty.
\end{align}

\item Translating this limit law back to $L$, we have that
\begin{align}
  \tau \simeq \frac{x}{2\alpha} + \sqrt{\frac{x}{8\alpha^3}} B,
\end{align}
with $B \sim N(0,1)$; since $L \sim \Exp(\tau)$,
\begin{align}
  \frac{x}{\sqrt{2\alpha}} L \simeq \Exp\left(1+\frac{4\alpha^2}{\sqrt{x}}B\right) \xrightarrow{x\to\infty} \Exp(1),
\end{align}
i.e.\ $L$ is approximately a Exponential with rate $\sqrt{2\alpha}/x$.

\item Note: if we wanted to kill the motion at some negative location (returns to original patch)
we'd use the result for BM killed at $-y$, say (Feller vol.2, XIV.5):
First note that since we kill at rate $\alpha$, the Laplace transform of $\tau$ is proportional to the Laplace transform of the first passage time,
shifted by $\alpha$, since $\E[e^{-\ell \tau}] \propto \int_0^\infty e^{-\alpha t} e^{-\ell t} p_x(t) dt$.
The Laplace transform of the first passage time from $0$ to $x$, killed on hitting $-y$, is
\begin{align}
 \frac{ e^{y\sqrt{2\ell}} - e^{-y\sqrt{2\ell}} } { e^{(x+y)\sqrt{2\ell}} - e^{-(x+y)\sqrt{2\ell}} } .
\end{align}
Note this reduces to the answer above ($e^{-x\sqrt{2\ell}}$) as $y \to \infty$.


\end{itemize}

\subsection{Two dimensions}

\begin{itemize}

\item In two dimensions, if everything is radially symmetric, we need to replace simple Brownian motion by a two-dimensional Bessel process.
Conditioning it to survive until $x$ is equivalent to giving it drift (cite?),
so this reduces to finding the Laplace transform of the hitting time of a diffusion.
I think this is in some Pitman \& Yor paper.

\item Radial symmetry implies a big annulus of habitat around the central patch.
If instead there are two smallish patches in two dimensions, 
we need only multiply the one-dimensional case by the probability that the $y$-coordinate is within the required range,
which is something involving the error function, or for large $x$,
is the width of the patch multiplied by $1/\sqrt{2 \pi \sigma^2 t}$.  
This replaces the $t^{3/2}$ in \eqref{eqn:arrival_time_density} by $t^2$, and back to G\&R, we have that the Laplace transform is in terms of $K_{1}$ instead of $K_{1/2}$;
in this case, we have that
\begin{align}
    \P\{ \tau < \infty \} &= \frac{x dy}{2\pi \sigma^4} \int_0^\infty t^{-2} \exp\left\{-\frac{x^2}{2\sigma^2 t} - \alpha t \right\} dt \\
                          &= \frac{x dy}{2\pi \sigma^4} 2 \left( \frac{ x^2 }{ 2 \alpha \sigma^2 } \right)^{-1/2} K_1\left(2 \sqrt\frac{x^2 \alpha}{2\sigma^2} \right) \\
                          &= \frac{dy\sqrt{2\alpha}}{\pi \sigma^3} K_1\left(\frac{x \sqrt{2\alpha}}{\sigma} \right)  \qquad \text{in } d=2
\end{align}
and we get $\E[e^{-\ell \tau_x}]$ by replacing $\alpha$ with $\alpha+\ell$.

% WRONGLY BASED ON 1/t INSTEAD OF 1/\sqrt{t}:
% and this is found in G\&R 8.468, $K_{3/2}(z) = \sqrt{\pi/2z} e^{-z} (1+1/z)$.
% The formula corresponding to equation \eqref{eqn:first_integral} above is then
% \begin{align}
%   \E[e^{-\ell \tau_x}] \propto \sqrt{\pi} \frac{2\sqrt{\ell+\alpha}}{x^2} e^{-x\sqrt{2(\ell+\alpha)}} \left( 1 + \frac{1}{x\sqrt{2(\ell+\alpha)}} \right) ,
% \end{align}
% and so
% \begin{align}
%   \E[e^{-\ell \tau_x}] &= \frac{\sqrt{\ell+\alpha}}{\sqrt{\alpha}} e^{-x\sqrt{2}\left(\sqrt{\ell+\alpha}-\sqrt{\alpha}\right)} \left( 1 + \frac{1}{x\sqrt{2(\ell+\alpha)}} \right) \left( 1 + \frac{1}{x\sqrt{2(\alpha)}} \right)^{-1} \\
%   &= \frac{x\sqrt{2(\ell+\alpha)}+1}{x\sqrt{2\alpha}+1} e^{-x\sqrt{2}\left(\sqrt{\ell+\alpha}-\sqrt{\alpha}\right)} .
% \end{align}
% 
% \item Differentiating the above, we get that $\E[\tau_x] = x^2 / (x\sqrt{2\alpha}+1) \sim x/\sqrt{2\alpha}$ and $\var[\tau_x] = x^3 / \sqrt{2\alpha}(x\sqrt{2\alpha}+1)^2 \sim x/(2\alpha)^{3/2}$; in fact, the same scaling produces the same limit law, since the $\ell/\sqrt{x}$ term dissappears.
% 
\end{itemize}



\subsection{$K$ and $\phi$}

From Jagers, section 2.6,
the mean and variance of $K$ are as follows.
Let the probability of $k$ offpsring be $p_k$ and let
\begin{align}
 f(u) =  \sum_k p_k u^k,
\end{align}
and $f_n$ the $n^\mathrm{th}$ composition of $f$.
Note that the mean number of offspring is $1-s_d = f'(1)$, and the variance is $\xi^2 = f''(1) + f'(1)(1-f'(1)) = f''(1) + s_d(1-s_d)$.
If $\sum_k p_k k \log k < \infty$, then 
\begin{align}
\phi = \lim_{n \to \infty} \frac{ \P\{ \mbox{survival to }n \}  }{ (1-s_d)^n } = \lim_{n \to \infty} \frac{ 1-f_n(0) }{ (1-s_d)^n } > 0 .
\end{align}
Furthermore,
\begin{gather*}
\E[K] = 1/\phi \qquad \mbox{and} \\
\E[K(K-1)] = \frac{ f''(1) }{ \phi s_d (1-s_d) } \qquad \mbox{so} \\
\var[K] = \frac{1}{\phi} \left( \frac{ \xi^2 }{ s_d (1-s_d) } + \frac{1}{1-\phi} - 1 \right) .
\end{gather*}
Note additionally that 
\begin{align}
    1 - \frac{1-f_n(u)}{1-f_n(0)} \searrow g(u) = \E[u^K] .
\end{align}

\subsection{Occupation density}

Suppose we have branching processes beginning at the origin at some constant rate $\lambda$ in time.
Then the occupation density at distance $x$, i.e.\ the expected number of particles found at location $x$ at any given time, at stationarity,
is $\lambda \int_0^\infty \E[Z_t(dx)] dt$, where $Z_t$ is the branching process.
Since spatial motion is independent of branching,
\begin{align}
  \E[Z_t(dx)] = e^{-\alpha t} \frac{1}{(2\pi \sigma^2 t)^{d/2}} \exp\left(-\frac{x^2}{2\sigma^2 t} \right) ,
\end{align}
and so using G\&R 3.471.9 and 8.468 again,
\begin{align} \label{eqn:bp_occupation}
  \int_0^\infty \E[Z_t(dx)] dt  &= \int_0^\infty e^{-\alpha t} \frac{1}{(2\pi \sigma^2 t)^{d/2}} \exp\left(-\frac{x^2}{2\sigma^2 t} \right) dt \\
  &= \begin{cases}
    \sqrt{ \frac{ 2 x }{ \pi \sigma^{3} \sqrt{2\alpha} } }  K_{1/2} \left(\frac{x \sqrt{2\alpha}}{\sigma}\right) =
    \frac{1}{\sigma \sqrt{2 \alpha}} \exp\left( - \frac{ x \sqrt{2\alpha} }{ \sigma } \right) \qquad & d=1 \\
    \frac{1}{\sigma^2 \pi} K_0\left(\frac{x \sqrt{2\alpha}}{\sigma}\right) 
    \sim \frac{ 1 }{ \sqrt{ \sigma^3  2\pi x \sqrt{2\alpha} } } \exp\left(-\frac{x\sqrt{2\alpha}}{\sigma}\right) \qquad & d=2  ,
\end{cases} 
\end{align}
where $K_0(z)$ is the modified Bessel function of the second kind of order 0.
Note that this is the Green function for a Brownian motion killed at rate $\alpha$.

We can combine this with the formula \eqref{eqn:eqfreq} for the equilibrium frequency $q(r) \approx C \exp( -\sqrt{2 \alpha} r / \sigma)$
to estimate $\lambda$, the effective outflux from the occupied patch: equating $\rho q(r)$ with $\lambda$ multiplied by equation \eqref{eqn:bp_occupation}, we get
\begin{align} \label{eqn:outflux_from_patch}
    \lambda = \text{(outflux from old patch)} = \begin{cases}
        C \rho \sigma \sqrt{2\alpha} \quad & d=1 \\
    C \rho \sigma^{3/2} (2\alpha)^{1/4} \sqrt{2 \pi x} \quad & d=2 
    \end{cases}
\end{align}
Note that this differs from the value $C\rho$ which we might have guessed.

\emph{The following is not making sense to me at the moment.} A more back-of-the-envelope calculation of this would be:
we know that the probability a branching process arrives at the patch is given by $(1/\mu_K)\exp(-x\sqrt{2 \alpha}/\sigma)$;
if it does it is present in $\mu_K$ copies, each of which leave a total of $1/s_m$ offspring.
To translate this to the above we would need the typical area covered by these (in some sense?).
% which must be around $\sigma \sqrt{2 \alpha}$.

\subsection{An integral}

For reference, by G\&R 3.471.9 and 8.468,
\begin{align}
  \int_0^\infty x^{n-1/2} e^{-\frac{\beta}{x}-\gamma x} dx &= 2 \left( \frac{\beta}{\gamma} \right)^{\frac{2n+1}{4}} K_{n+1/2}(2\sqrt{\beta\gamma}) \quad \mbox{(for general $n$)} \\
  \mbox{($n$ an integer)} \quad &= \sqrt{\pi} \beta^{\frac{n}{2}} \gamma^{-\frac{n+1}{2}} e^{-2\sqrt{\beta\gamma}} \sum_{k=1}^n \frac{ (n+k)! }{ k! (n-k)! (4\sqrt{\beta \gamma})^k } .
\end{align}

Note that (8.451.6) $K_\nu(z) = \sqrt{\frac{\pi}{2z}} e^{-z} (1 + O(1/z))$ for all $\nu$, and $K_{-\nu}(z)=K_\nu(z)$.

We have $n-1/2 = - d/2$, $\beta=x^2/2\sigma^2$, and $\gamma = \alpha$, leading to
\begin{align}
\frac{1}{(2\pi\sigma^2 )^{d/2}} \int_0^\infty x^{n-1/2} e^{-\frac{\beta}{x}-\gamma x} dx \sim
\frac{ 1 }{ (2\pi)^\frac{d-1}{2} \sigma^d } \left( \frac{\sigma}{x} \right)^{\frac{d-1}{2}} ({2\alpha})^\frac{3-d}{4} \exp\left( - \frac{ x \sqrt{2 \alpha} }{ \sigma } \right) \quad \mbox{as}\; \frac{x\sqrt{2\alpha}}{\sigma} \to \infty .
\end{align}

%%%%%%%%%
\section{The big picture}

Here are notes to clarify what's going on for writing the main paper.

From the main paper, we have that
\begin{align} \label{eqn:bp_prob_estab}
    \text{(prob of estab of a distant BP)} &\approx \frac{2s_b}{\xi^2} \exp\left( - \frac{\sqrt{2 \alpha} x}{\sigma} \right) ,
\end{align}
and the subcritical occupation density per unit area at distance $r$ is
\begin{align} \label{eqn:eqfreq}
    q(r) \approx C \exp( -\sqrt{2 \alpha} r / \sigma).
\end{align}

We can calculate the rate of ``core'' establishment, i.e.\ the mean number of reproductive events that lead to establishment as defined below, in two ways:
\begin{align}
    (\mbox{rate of establishment}) &= (\mbox{outflux of BPs from old patch}) \times (\mbox{\# estab events per BP}) \label{eqn:heuristic1a} \\
    \begin{split} &= (\mbox{subcritical occupation density}) \times \rho \times (\mbox{\# offspring per subcritical indiv}) \label{eqn:heuristic1b} \\
    & \qquad \qquad \times (\mbox{prob of estab in the patch}) . \end{split} 
\end{align}
The subcritical occupation density is the integral of equation~\eqref{eqn:eqfreq} over the patch;
the number of (extra) offspring per subcritical individual is $s_B+s_m$;
and the probability of establishment of each such extra offspring in the patch is $2s_B/\xi^2$.
The effective outflux of branching processes is computed in equation \eqref{eqn:outflux_from_patch}.
The number of establishment events per BP is as follows, in averages: 
with prob $1/\mu_K \exp(-\sqrt{\alpha} x/ \sigma)$ a total of $\mu_K$ indivs arrive at the new patch; 
each produce $1/s_m$ total core offspring;
each of these produce $s_B+s_m$ extra offspring which each have chance $2s_B/\xi^2$ of establishing,
i.e.
\begin{align}
  (\mbox{\# estab events per BP}) = \frac{(s_B+s_m) 2 s_B }{ s_m \xi^2 } \exp\left(-\frac{\sqrt{2\alpha} x}{\sigma}\right) .
\end{align}

However, this formulation counts separately each of possible multiple establishment events by a single transiting family.
Writing
\begin{align} \label{eqn:heuristic2}
  \begin{split} (\mbox{rate of establishment}) &= (\mbox{outflux of BPs from old patch}) \times (\mbox{prob of estab of a distant BP}) \\
  & \qquad \qquad \times (\mbox{\# estab events per successful BP}) \end{split},
\end{align}
for $\migrate$ we actually want to omit the last term.
Using the probability of establishment of a distant branching process is in equation~\eqref{eqn:bp_prob_estab}, this is
\begin{align}
  \migrate &= (\mbox{outflux of BPs from old patch}) \times (\mbox{prob of estab of a distant BP}) \\
           &= \begin{cases}
    C \rho \sqrt{2 \alpha} \times \frac{2 s_b}{\sqrt{\pi} \xi^2} \exp\left(- \frac{\sqrt{2\alpha}x}{\sigma} \right)  \qquad & d=1 \\
    C \rho \sqrt{2 \alpha} \times \frac{ 2 s_b \sqrt{2 \alpha A} }{ \xi^2 \pi \sigma^3} K_1\left(\frac{x\sqrt{2\alpha}}{\sigma}\right)  \qquad & d=2 .
\end{cases}
\end{align}
Finally, combining \eqref{eqn:heuristic1b} and \eqref{eqn:heuristic2}, we can also compute that
\begin{align}
  (\mbox{\# estab events per successful BP}) = \frac{s_B+s_m}{ \sqrt{s_m} } .
\end{align}

\end{document}
