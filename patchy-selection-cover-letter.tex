\documentclass{article}
\usepackage{natbib}
\usepackage{fullpage}
\usepackage{hyperref}

\begin{document}

% \section{Cover letter}

\begin{minipage}[b]{2.5in}
  Submission Cover Letter \\
  {\it PLoS Genetics}
\end{minipage}
\hfill
\begin{minipage}[b]{2.5in}
  Departments of Evolution and Ecology\\
  University of California at Davis\\
  \textbf{and} \\
  Molecular Biology and Bioinformatics \\
  University of Southern California\\
  \today
\end{minipage}

\vskip 2em

\noindent
{\bf To the Editor(s) -- }

\vskip 1em
Please find enclosed our manuscript, entitled 
``Convergent Evolution During Local Adaptation to Patchy Landscapes'',
for consideration for publication in {\it PLoS Genetics}.

\vskip 1em

In the paper, we study the process of adaptation on a continuous landscape 
with patches of habitat in which a selective pressure is contrary to that experienced in the rest of the range,
determining the quantities that describe how adaptation will proceed.
The work is theoretical, with an empirical example.
We strove to present the substantial mathematical results in a clear manner accessible to the readership of {\it PLoS Genetics},
focusing instead on the biological intuition and interpretations.
The paper is in the same vein as our well-read paper 
``Parallel adaptation: one or many waves of advance of an advantageous allele?'' (Genetics 2010),
and fills a gap in empirical workers' toolboxes and intuitions surrounding the geographic aspects of adaptation.


\vskip 1em

The work is motivated by the classical examples of intraspecific convergent evolution across patchy environments,
such as heavy metal tolerance in plants, or locally adaptive melanism in small mammals.
Such examples are of considerable interest because they 
imply a certain degree of constraint imposed on evolution by the genomic architecture underlying the trait.
Over the coming decade a large number of population genomics datasets
with fine-grain geographic sampling will be developed, these will likely highlight many cases of geographical convergent
evolution. Our model helps build intuition about the scale at which we
should expect such convergent evolution among environmental patches,
and so may provide an important reference point in guiding the design of
these studies. 


\vskip 1em

We suggest as suitable reviewers: 
Joachim Hermisson (\href{mailto:joachim.hermisson@univie.ac.at}{joachim.hermisson@univie.ac.at}), 
Pleuni Pennings (\href{mailto:pleuni@dds.nl}{pleuni@dds.nl}),
% Magnus Nordborg (\href{mailto:magnus@usc.edu}{magnus@usc.edu}),
Phillip Messer (\href{mailto:messer@cornell.edu}{messer@cornell.edu}),
Mark Kirkpatrick (\href{mailto:kirkp@austin.utexas.edu}{kirkp@austin.utexas.edu}),
and Nick Barton
(\href{mailto:nick.barton@ist.ac.at}{nick.barton@ist.ac.at}). Any of
these reviewers may also be a suitable external editor (as Coop is an
associate editor). 


\vskip 2em

\noindent \hspace{4em}
\begin{minipage}{3in}
\noindent
{\bf Sincerely,}

\vskip 2em

{\bf Peter Ralph, and\\
Graham Coop.}\\
\end{minipage}

\end{document}

