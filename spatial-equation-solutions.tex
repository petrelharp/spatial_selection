\documentclass{article}
\usepackage{amsmath,amssymb}
\usepackage{fullpage}

\begin{document}

Slatkin (1973) argues that if selection at $x$ is $s\gamma(x)$,
the variance in dispersal distance is $\sigma^2$,
and $A(x) = 2p(x)-1 = p(x)-q(x)$, then
\begin{equation}
    \partial_x^2 A(x) = - \frac{s \gamma(x)}{\sigma^2} \left( 1- A^2(x) \right).
\end{equation}
(note the erroneous minus sign in equation (7).
Rescaling to $\xi = x \sqrt{s}/\sigma$, this is
\begin{equation} \label{eqn:reaction_diffusion}
    \partial_\xi^2 A(\xi) = - \gamma(x) \left( 1- A^2(\xi) \right).
\end{equation}

Note that if $f(x) = 2 - 3 \tanh^2(x/2+x_0)$ then since $\partial_x \tanh(x) = (1-\tanh^2(x))$,
\begin{align*}
    2 &\ge f(x) \ge -1 \\
    \partial_x f(x) &= - 3 \tanh(x/2+x_0) \left( 1- \tanh^2(x/2+x_0) \right) \\
    \partial^2_x f(x) &= - \frac{3}{2} \left( 1- \tanh^2(x/2+x_0) \right) 
    + \frac{9}{2} \tanh^2(x/2+x_0) \left(1-\tanh^2(x/2+x_0)\right) \\
        &= \frac{1}{2} \left\{ - 3 + 12 \tanh^2(x/2+x_0) - 9 \tanh^4(x/2+x_0) \right\} \\
        &= \frac{1}{2} \left(1-f(x)^2 \right)
\end{align*}
and a solution to equation \eqref{eqn:reaction_diffusion}, if $\gamma(x)=\gamma$ is constant,
is obtained by choosing $x_0$ to match the boundary conditions.

Note that since $-1\le A(x) \le 1$, we certainly can't apply the solution $f(x)$
for $|x/2+x_0| \le \tanh^{-1}(1/\sqrt{3})$.


%%%%%%%%%%
\section{The probability of establishment}
\label{ss:prob_estab}


In the main text we have made use of two decompositons of the spatial branching process
to approximate in two different ways the rate of establishment of migrant alleles.
Here we make the relationship more formal.
The decompositions use a similar idea to that found in \citep[section D.12]{athreya2004branching},
where a supercritical branching process is decomposed into individuals whose families eventually die out
and those having infinite lines of descent.
We also note that the approximations fall generally in the realm covered by \citet{aldous1989poisson}.

Consider a discrete-time branching process $Z_t$ with spatial motion and offspring distribution depending on spatial location.
Suppose that the offspring distribution for an individual at location $x$ is distributed as $X_x$.
We can record the state of the process simply as the list of locations of individuals:
$Z_t = (x_1, \ldots, x_{N(t)})$ means that there are $N(t)$ individuals alive at $t$, at locations $x_1, \ldots, x_{N(t)}$.
We can construct the process iteratively: given $Z_t$,
we sample first the number of offspring of each, and then the locations the offspring migrate to.
Concretely, given $Z_{t} = (x_1, \ldots, x_{N(t)})$,
first let $X_1(t),\ldots,X_{N(t)(t)}$ be independent random draws from the offspring distribution,
with $X_k(t)$ having distribution $X_{x_k}$.
The number of new offspring is $N(t+1) = \sum_{k=1}^{N(t)} X_k(t)$,
and we can sample their locations independently given the location of their parent to obtain $Z_{t+1}$.
Note that the genealogy is implicitly recorded:
the parent of the $k^\mathrm{th}$ individual at time $t$ is
the $a(k,t) = \min\{ \ell : \sum_{j=1}^\ell X_j(t-1) \ge k \}$-th individual at time $t-1$.
It will be useful to extend this to more distant ancestors:
let $a_1(k,t) = a(k,t)$ and recursively define $a_m(k,t) = a(a_{m-1}(k,t),t-m+1)$, as long as $m\le t$.

We want to find (approximations to) $p(x)$, the probability that a branching process beginning with a single individual at $x$ does not die out. 
(In other words, $p(x) = \P\{ N(t) = 0 \; \mbox{for some}\; t>0 \; \vert \; Z_0 = (x) \}$.)
We assume that the mean number of offspring $\E[X_x]$ is only greater than one for $x \in A$ (``the patch''),
and so the process can only establish if some offspring eventually get to the patch (with probability one).

The second decomposition we use is simpler, only conditioning on the number of individuals who ever reach the patch.  
Let $\calE = \{(t_i,k_i)\}$ be the times and indices of individuals who are in the patch but none of whose ancestors have been --
i.e.\ $(t,k) \in \calE$ iff $x_k(t) \in A$ and $a_m(k,t) \notin A$ for all $1\le m \le t$.
If we remove these individuals (and their offspring), the resulting branching process has mean number of offspring less than or equal to 1 everywhere,
so there are a finite number of such individuals with probability one (the process dies out).
Therefore, the probability that the process dies out is then the probability that the families descending from each individual in $\calE$ dies out.
By independence of different offspring, this is
\begin{align}
  1-p(x) = \E\left[ \prod_{(t,k) \in \calE} (1-p(x_k(t))) \; \vert \; Z_0 = (x) \right] .
\end{align}
Above, we make the approximation that $p(x_k(t)) \approx p_e$,
the probability of establishment ignoring the possibility of migrating outside the patch.
In this case,
\begin{align}
  p(x) \approx \E\left[ p_e^{|\calE|} \; \vert \; Z_0 = (x) \right] .
\end{align}
Here $|\calE|$ is the ``size of the family that reaches the patch'';
in section \ref{ss:haplotype_length} we estimated the probability that any of the family reaches the patch
and argued that the distribution of $|\calE|$ is given by $K$,
the size of the subcritical branching process conditioned on still being alive.


The first approximation uses a similar trick, but rather than treating all individuals migrating into the patch
as an opportunity for establishment,
we couple the branching process to a strictly subcritical branching process
that produces extra offspring, each of which provides an opportunity for establishment.
To this end, suppose that the offspring distribution off the patch is given by $X'$ (so $X_x \deq X'$ for $x \notin A$),
and the offspring distribution within the patch stochastically dominates $X$,
i.e.\ $\P\{X_x\ge n\} \ge \P\{X'\ge n\}$ for all $x\in A$ and $n \ge 0$.
Stochastic domination implies that we can \emph{couple} any instance of $X_x$ with one of $X'$ and a remainder --
i.e.\ jointly construct an $X'$ and a $Y_x$ such that $X_x = X' + Y_x$,
and that $Y_x$ also takes values in the nonnegative integers.
We can therefore modify the construction of the branching process above,
so that $X_k(t) = X_k'(t) + Y_k(t)$, where $X_k'(t) \deq X$ and $Y_k(t) \deq Y_{x_k(t)}$.
We can define a \emph{core} branching process $Z'$ using the $X'$, which will be finite (since $\E[X']<1$).
Formally, say that the first $X_k'(t)$ offspring of the $(t,k)^\mathrm{th}$ individual are ``core'' offspring,
denoted by $\calC_t = \{ (t,k) \; : \; 0 \le k - \sum_{j<\ell} X_j(t) \le X_\ell'(t) \; \mbox{for some}\; 1 \le \ell \le N(t) \}$;
and define $Z'$ as the core individuals whose ancestors are all ``core'' as well,
i.e.\ $Z'_t = \{ x_k \in Z_t \; : \;  a_m(k,t) \in \calC_{t-m} \; \mbox{for all}\; 0 \le m \le t\}$.
The boundary analogous to above in the first computation is then the non-core offspring of core individuals;
denote the times and indices of these by $\partial \calC$,
so that $(t,k) \in \partial \calC$ iff $(t,k) \notin \calC_t$ and $a(k,t) \in \calC_{t-1}$.
We can now use this as above to compute the probability of establishment beginning from a single individual:
\begin{align}
  1-p(x) = \E\left[ \prod_{(t,k) \in \partial\calC} (1-p(x_k(t))) \; \vert \; Z_0 = (x) \right] ,
\end{align}
but this is less helpful, as the distribution of $\calC$ is harder to get at.

However, recall the actual problem at hand.
Roughly speaking, we approximate the process of colonization of the new patch by migrants from an established patch
by imagining that the established patch sends out migrants, and that each of these acts as an independent branching process,
at least in the region of the new patch.
We can suppose that migrant families appear at some rate, say, midway between the patches, and far enough from the original one that we can ignore it,
and leave the precise distribution of their appearance unspecified
We can decompose each of these branching processes as above into ``core'' and ``extra'',
so that the point in time of ``establishment'' is well-defined:
the time $\tau$ of first appearance of an ``extra'' offspring with infinite line of descent.
Given that establishment has not yet occurred by $t$, 
and the combined state of the core process $Z'_t$,
then the mean number of establishment events occurring at time $t$ is $\sum_{x \in Z'_t} p(x) \E[Y_x]$.

If establishment events were independent and sufficiently rare,
the influx of branching processes begain long enough ago that $Z'$ was at stationarity
with $z_A := \E[ \# \{ (x,t) \in Z'_t : x \in A \} ]$,
and we again approximate $p(x) \approx p_e$,
then we would have that 
\begin{align}
  \P\{\tau>t\} \approx \exp\left\{ - t z_A \E[Y_x] p_e \right\} ;
\end{align}
in our case, $\E[Y_x] \approx (s_p+s_m)$ and $p_e \approx 2 s_p/\xi^2$,
so $\tau$ is approximately exponentially distributed with rate $ 2 z_A (s_p+s_m) s_p/\xi^2 $.
We can get $z_A$, the occupation density without the new patch, 
directly by integrating the expression~\eqref{eqn:eqfreq} over the patch.

In obtaining an exponential distribution, we have assumed effectively that establishment events
are independent of each other.
This will not be strictly true, since an establishment implies that a family of migrants are present in the patch,
and hence others may produce offspring with infinite lines of descent at a similar time.
The mean total family size of an allele with fitness $1-s_m$ (i.e.\ a subcritical branching process) is $\sum_{t \ge 0} (1-s_m)^t = 1/s_m$;
so the mean number of ``extra'' offspring is $1+s_p/s_m$.
Establishment events are correlated to the extent that single such families give rise to multiple establishment events,



% We will record the state of the branching process as a finite point measure,
% so that if there are $n$ individuals alive at $t$, at locations $x_1, \ldots, x_n$,
% then $Z_t = \sum_{k=1}^n \delta_{x_k}$, and we write $Z_t(A) = \sum_{k=1}^n \bone_{x_k \in A}$ for the number of individuals in a set $A$.



\end{document}
