\documentclass{article}

\usepackage{fullpage}
\usepackage{amsmath}
\usepackage{amssymb}

\newcommand{\E}{\mathbb{E}}
\renewcommand{\P}{\mathbb{P}}
\newcommand{\deriv}[1]{\frac{d}{d#1}}
\newcommand{\dderiv}[1]{\frac{d^2}{d^2#1}}
\newcommand{\given}{\;\vert\;}

\begin{document}

\section{Discrete model}

Consider a discrete model in which the total number of individuals at location $x$ is $N(x)$,
each individual is of type either B or b at a selected locus,
and the number of individuals of type B and location $x$ is $n(x)$.
Suppose that type B individuals at location $x$ reproduce at rate $1+s_B(x)$, 
and likewise type b at rate $1+s_b(x)$.
At reproduction, individuals recombine with others in the same location,
and the offspring choose a new location according to the kernel $m(x,y)$.
The population dynamics are random, but suppose that $N(x)$ is sufficiently large that these do not vary with time;
instead we want to follow a lineage backwards in time,
at a locus separated from the selected locus by recombination fraction $r$.

Suppose this is a Moran model.
There are four things that can happen:
\begin{enumerate}
    \item[$x\xrightarrow{AA}y$] One type B individual at location $x$ reproduces, 
        either does not recombine or recombines with another type B,
        and sends the offspring to $y$.
    \item[$x\xrightarrow{AB}y$] An individual at location $x$ reproduces, 
        recombines with the other type,
        and sends to $y$ an offspring
        who inherits at the selected locus from the type b parent 
        and the at the neutral locus from the type B parent.
    \item[$x\xrightarrow{BA}y$] An individual at location $x$ reproduces, 
        recombines with the other type,
        and sends to $y$ an offspring
        who inherits at the selected locus from the type B parent 
        and the at the neutral locus from the type b parent.
    \item[$x\xrightarrow{BB}y$] One type b individual at location $x$ reproduces, 
        either does not recombine or recombines with another type b,
        and sends the offspring to $y$.
\end{enumerate}
Let $p(x) = n(x)/N(x)$.
These four things happen at rates:
\begin{align}
    & x\xrightarrow{AA}y & \qquad w_{AA}(x,y) &= p(y) (1+s_B(y)) \left(1 - r (1-p(y)) \right)  N(y) m(y,x) \\
    & x\xrightarrow{AB}y & \qquad w_{AB}(x,y) &= r p(y) (1-p(y)) \left(1+\frac{s_B(y)+s_b(y)}{2}\right) N(y) m(y,x) \\
    & x\xrightarrow{BA}y & \qquad w_{BA}(x,y) &= r p(y) (1-p(y)) \left(1+\frac{s_B(y)+s_b(y)}{2}\right) N(y) m(y,x) \\
    & x\xrightarrow{BB}y & \qquad w_{BB}(x,y) &= (1-p(y)) (1+s_b(y)) \left(1 - r p(y) \right) N(y) m(y,x) 
\end{align}

Note that at equilibrium, we require
\begin{align}
  0 = \sum_y N(y) m(y,x) \left\{ p(y)(1+s_B(y))(1-p(x)) - (1-p(y))(1+s_b(y))p(x) \right\} .
\end{align}

\subsection{Lineage movement}

These rates tell us the rates at which a lineage will move, backwards in time.
For instance, the rate at which a lineage at the selected locus
currently in a type B individual at location $x$
jumps to another type B individual at location $y$ is equal to the rate of influx of migrants from $y$
divided by the number of B alleles at $x$,
or
\begin{align}
  r_A(x,y) = \frac{w_AA(y,x)}{p(x)N(x)} = N(y) p(y)  (1+s_B(y)) m(y,x) \frac{ 1 }{ N(x) p(x) }
\end{align}


A lineage at the selected locus (with $r=0$) in a type B individual moves according to these jump rates.
If we follow a linked locus until the first time it is found in a type b individual,
it moves according to these jump rates, and at location $x$ recombines at rate
\begin{align}
    \gamma_B(x) = \sum_y w_{BA}(y,x) .
\end{align}
Suppose that we kill the lineage when it recombines onto type b.
Let $X_t$ denote the position of the lineage at time $t$ in the past,
with $X_t = \rho$ if it has recombined onto b,
and let $f$ be test function with $f(\rho)=0$.
If we let $M_r(y) = p(y) (1+s_B(y)) \left(1 - r (1-p(y)) \right)  N(y)$, then
\begin{align}
    \deriv{t} \E[f(X_t) \given X_0=x ] &= \sum_y w_{AA}(y,x) ( f(y)-f(x) ) - \gamma_B(x) f(x) \\
                                       &= \sum_y  M_r(y) m(y,x) ( f(y) - f(x)  ) - \gamma_B(x) f(x) .
\end{align}

\section{Diffusion limit}

Now suppose that $m(x,y)$ is symmetric, and depends on a parameter $\sigma$ so that as $\sigma \to 0$,
the associate random walk converges to Brownian motion, so that
\begin{align}
    \lim_{\sigma \to 0} \sum_y \frac{ m(x,y) ( f(y) - f(x) ) }{\sigma^2} = \frac{1}{2} \dderiv{x} f(x) .
\end{align}
Write $f'(x) = \deriv{x}f(x)$, and note that
\begin{align}
    \frac{1}{\sigma^2} \sum_y g(y) m(y,x) (f(y)-f(x)) 
    &= \frac{1}{\sigma^2} \sum_y m(y,x) \left( g(y) f(y) - g(x) f(x) + (g(x)-g(y)) f(x) \right) \\
    &= \frac{1}{\sigma^2} \sum_y m(y,x) \left( g(y) f(y) - g(x) f(x) \right) \\
    & \qquad - f(x) \frac{1}{\sigma^2} \sum_y m(y,x) (g(y)-g(x)) \\
    &\xrightarrow{\sigma \to 0} \frac{1}{2} \dderiv{x}\left( g(x)f(x) \right) - \frac{1}{2} f(x) \dderiv{x} g(x) \\
    &= \frac{1}{2} \left( g(x) f'(x) + 2 g'(x) f'(x) + f(x) g''(x) - f(x) g''(x) \right) \\
    &= \frac{1}{2} g(x) f''(x) + g'(x) f'(x) .
\end{align}
If we also rescale recombination, so that $\rho = r/\sigma^2$,
then 
\begin{align}
    \frac{1}{\sigma^2} \gamma_B(x) \xrightarrow{\sigma \to 0} \rho p(x) (1-p(x)) \left( 1 + \frac{s_B(x)+s_b(x)}{2} \right) N(x) ,
\end{align}
which we can define to be equal to $\rho k(x)$.
In this case, note that since $r = \rho \sigma^2$,
\begin{align}
    M_r(x) \xrightarrow{\sigma \to 0} M(x) = p(x) (1+s_B(x)) N(x) .
\end{align}
Under these assumptions,
\begin{align}
    \deriv{t} \E[f(X_{t/\sigma^2}) \given X_0=x ] &\xrightarrow{\sigma \to 0} 
    \frac{1}{2} M(x) f''(x) + M'(x) f'(x) - \rho k(x) f(x) ,
\end{align}
i.e.\ $X_{t/\sigma^2}$ converges to a diffusion with drift $M'(x)$ and killed at rate $\rho k(x)$.

\paragraph{Stationary distribution}
If $X$ has a stationary distribution then its density $\pi(x)$ solves the forward Kolmogorov equation,
\begin{align}
    0 &= - \deriv{x}\left( M'(x) \pi(x) \right) + \frac{1}{2} \dderiv{x} \left( M(x) \pi(x) \right) \\
      &= - M''(x) \pi(x) - M'(x) \pi'(x) + \frac{1}{2} M''(x) \pi(x) + M'(x) \pi'(x) + \frac{1}{2} M(x) \pi''(x)  \\
      &= \frac{1}{2} \left( M(x) \pi''(x) - M''(x) \pi(x) \right) .
\end{align}
It is immediate that $\pi(x) = M(x)/Z$,
where
\begin{align}
    Z = \int p(x) (1+s_B(x)) N(x) dx .
\end{align}

\paragraph{Recombination time}
Let $\tau_\rho$ be the first time  the lineage recombines onto type b.
If $\rho$ is much smaller than the mixing time of a type B lineage,
then the time to recombination is roughly Exponential,
with mean
\begin{align}
    \E[\tau_\rho] \approx \frac{1}{ \rho \int \pi(x) k(x) dx } .
\end{align}
The product $\pi(x) k(x)$ is:
\begin{align}
    \pi(x) k(x) = \frac{1}{Z} N(x)^2 p(x)^2 (1-p(x)) (1+s_B(x)) (1+(s_B(x)+s_b(x))/2) .
\end{align}

More generally, if we define
\begin{align}
    h(x) = \E[\tau_\rho \given X_0 = x]
\end{align}
then $h$ solves the equation
\begin{align}
    -1 = M'(x) h'(x) + \frac{1}{2} M(x) h''(x) - \rho k(x) h(x) .
\end{align}

The probability that a segment of length $r = \rho \sigma^2$ to one side of the selected locus
does not recombine across time $t$ is $\P\{\tau_\rho > t\}$.
Therefore, if $L_t$ is the length of the nonrecombined segment to one side of the selected locus
after time $t$, then
\begin{align}
    \E[ L_t \given X_0=x ] &= \sigma^2 \int_0^\infty \P\{ \tau_\rho > t \} d\rho \\
\end{align}

The appropriate thing to do is to look at $\E[L_T]$, where $T$ is the coalescence time.
Generally, then
\begin{align}
  \P\{ L_t > \rho \} = \P\{ \tau_\rho > T \} .
\end{align}
Do do this properly, we need to run two diffusions.
However, if we let $T$ instead be independent Exponential($\lambda$), then 
\begin{align}
  \ell(x) = \P\{ \tau_\rho > T \given X_0 = x\} 
\end{align}
solves the equation
\begin{align}
  M'(x) \ell'(x) + \frac{1}{2} M(x) \ell''(x) - (\lambda + \rho k(x) ) \ell(x) = \lambda .
\end{align}

\paragraph{Questions:} 
\begin{enumerate}

    \item Can we make any progress analytically with any of these quantities,
      using the fact that
      \[ \frac{1}{2} p''(x) + s(x) p(x)(1-p(x)) = 0 \quad ?\]

    \item Do we get the same system using, say, a Wright--Fisher model?  (I assume so?)

    \item Can we prove convergence of the finite model to this one, under suitable assumptions?  A main difficulty here would be showing that the lineage spends all its time in the part of the range where $p(x)$ does not fluctuate significantly, i.e.\ is not too small\ldots but if this is not true, it would be even more interesting.

\end{enumerate}

\end{document}
