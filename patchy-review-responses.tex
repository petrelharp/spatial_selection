% \documentclass{article}
% \usepackage{fullpage}
% \usepackage{lineno}
% \usepackage[hypertexnames=false]{hyperref}   % not working correctly
%\usepackage{latexml}

\linenumbers


%%%%%  PUT THIS IN HEADER OF FILE
% % Responses to reviews:
% \usepackage{lineno}
% \usepackage[hypertexnames=false]{hyperref}   % not working correctly
%\usepackage{latexml}

\linenumbers


%%%%%  PUT THIS IN HEADER OF FILE
% % Responses to reviews:
% \usepackage{lineno}
% \usepackage[hypertexnames=false]{hyperref}   % not working correctly
%\usepackage{latexml}

\linenumbers


%%%%%  PUT THIS IN HEADER OF FILE
% % Responses to reviews:
% \input{review-response-commands}
% % set this to show line numbers and include responses to reviews or not
% \newif\ifreviewresponses
% \reviewresponsestrue  % include them
% % \reviewresponsesfalse  % don't include them
% \newcommand{\responsefile}{pbio-reviews-19sept12-responses.tex}  % name of the review reponses file

% counters for reviewer points
\newcounter{reviewer}
\setcounter{reviewer}{0}
\newcounter{point}
\setcounter{point}{0}

\newcommand{\reviewersection}{\refstepcounter{reviewer}\setcounter{point}{0}
                  \section*{Reviewer \#\thereviewer:}}
% drawing from from http://tex.stackexchange.com/questions/2317/latex-style-or-macro-for-detailed-response-to-referee-report
\newcommand{\point}[2]{\refstepcounter{point} \bigskip \hrule \medskip \noindent 
        \slshape {\fontseries{b}\selectfont (\thereviewer.\thepoint) #1} #2} 
\newcommand{\reply}{\normalfont \medskip \noindent \textbf{Reply}:\ }   

% use this command in the text where a change addressing a reviewer point has occurred
\newcommand{\revpoint}[2]{\linelabel{rr:rev#1:#2}}
% and this one to refer to such a location
\newcommand{\revref}{{\iflatexml{}\else{(p.\ \pageref{rr:rev\thereviewer:\thepoint}, l.\ \lineref{rr:rev\thereviewer:\thepoint})}\fi}}

% or, this one to refer to a named linelabel
\newcommand{\llname}[1]{{\iflatexml{}\else{(p.\ \pageref{#1}, l.\ \lineref{#1})}\fi}}

% put this where the reviews are to appear (at the end?)
\newcommand{\includereviews}{
    \ifreviewresponses
    \clearpage
    \setcounter{page}{1}
    \setcounter{section}{0}
    \setcounter{subsection}{0}
    \nolinenumbers
    \begin{center}
      {\LARGE \bf Response to Reviews}
    \end{center}
    \input{\responsefile}
    \fi
}

% Useful shortcuts
\newcommand{\rollover}{ \reply{The reviewer makes an excellent point that we have missed out entirely.  We have made all the changes suggested, down to the minutiae \revref.} }
\newcommand{\playdead}{ \reply{The reviewer makes an excellent point.  We have made an utterly trivial change {\revref} that we think deals entirely with the concern raised.} }


% from http://tex.stackexchange.com/questions/43648/why-doesnt-lineno-number-a-paragraph-when-it-is-followed-by-an-align-equation/55297#55297
\newcommand*\patchAmsMathEnvironmentForLineno[1]{%
  \expandafter\let\csname old#1\expandafter\endcsname\csname #1\endcsname
  \expandafter\let\csname oldend#1\expandafter\endcsname\csname end#1\endcsname
  \renewenvironment{#1}%
     {\linenomath\csname old#1\endcsname}%
     {\csname oldend#1\endcsname\endlinenomath}}% 
\newcommand*\patchBothAmsMathEnvironmentsForLineno[1]{%
  \patchAmsMathEnvironmentForLineno{#1}%
  \patchAmsMathEnvironmentForLineno{#1*}}%
\AtBeginDocument{%
\patchBothAmsMathEnvironmentsForLineno{equation}%
\patchBothAmsMathEnvironmentsForLineno{align}%
\patchBothAmsMathEnvironmentsForLineno{flalign}%
\patchBothAmsMathEnvironmentsForLineno{alignat}%
\patchBothAmsMathEnvironmentsForLineno{gather}%
\patchBothAmsMathEnvironmentsForLineno{multline}%
}

% % set this to show line numbers and include responses to reviews or not
% \newif\ifreviewresponses
% \reviewresponsestrue  % include them
% % \reviewresponsesfalse  % don't include them
% \newcommand{\responsefile}{pbio-reviews-19sept12-responses.tex}  % name of the review reponses file

% counters for reviewer points
\newcounter{reviewer}
\setcounter{reviewer}{0}
\newcounter{point}
\setcounter{point}{0}

\newcommand{\reviewersection}{\refstepcounter{reviewer}\setcounter{point}{0}
                  \section*{Reviewer \#\thereviewer:}}
% drawing from from http://tex.stackexchange.com/questions/2317/latex-style-or-macro-for-detailed-response-to-referee-report
\newcommand{\point}[2]{\refstepcounter{point} \bigskip \hrule \medskip \noindent 
        \slshape {\fontseries{b}\selectfont (\thereviewer.\thepoint) #1} #2} 
\newcommand{\reply}{\normalfont \medskip \noindent \textbf{Reply}:\ }   

% use this command in the text where a change addressing a reviewer point has occurred
\newcommand{\revpoint}[2]{\linelabel{rr:rev#1:#2}}
% and this one to refer to such a location
\newcommand{\revref}{{\iflatexml{}\else{(p.\ \pageref{rr:rev\thereviewer:\thepoint}, l.\ \lineref{rr:rev\thereviewer:\thepoint})}\fi}}

% or, this one to refer to a named linelabel
\newcommand{\llname}[1]{{\iflatexml{}\else{(p.\ \pageref{#1}, l.\ \lineref{#1})}\fi}}

% put this where the reviews are to appear (at the end?)
\newcommand{\includereviews}{
    \ifreviewresponses
    \clearpage
    \setcounter{page}{1}
    \setcounter{section}{0}
    \setcounter{subsection}{0}
    \nolinenumbers
    \begin{center}
      {\LARGE \bf Response to Reviews}
    \end{center}
    \input{\responsefile}
    \fi
}

% Useful shortcuts
\newcommand{\rollover}{ \reply{The reviewer makes an excellent point that we have missed out entirely.  We have made all the changes suggested, down to the minutiae \revref.} }
\newcommand{\playdead}{ \reply{The reviewer makes an excellent point.  We have made an utterly trivial change {\revref} that we think deals entirely with the concern raised.} }


% from http://tex.stackexchange.com/questions/43648/why-doesnt-lineno-number-a-paragraph-when-it-is-followed-by-an-align-equation/55297#55297
\newcommand*\patchAmsMathEnvironmentForLineno[1]{%
  \expandafter\let\csname old#1\expandafter\endcsname\csname #1\endcsname
  \expandafter\let\csname oldend#1\expandafter\endcsname\csname end#1\endcsname
  \renewenvironment{#1}%
     {\linenomath\csname old#1\endcsname}%
     {\csname oldend#1\endcsname\endlinenomath}}% 
\newcommand*\patchBothAmsMathEnvironmentsForLineno[1]{%
  \patchAmsMathEnvironmentForLineno{#1}%
  \patchAmsMathEnvironmentForLineno{#1*}}%
\AtBeginDocument{%
\patchBothAmsMathEnvironmentsForLineno{equation}%
\patchBothAmsMathEnvironmentsForLineno{align}%
\patchBothAmsMathEnvironmentsForLineno{flalign}%
\patchBothAmsMathEnvironmentsForLineno{alignat}%
\patchBothAmsMathEnvironmentsForLineno{gather}%
\patchBothAmsMathEnvironmentsForLineno{multline}%
}

% % set this to show line numbers and include responses to reviews or not
% \newif\ifreviewresponses
% \reviewresponsestrue  % include them
% % \reviewresponsesfalse  % don't include them
% \newcommand{\responsefile}{pbio-reviews-19sept12-responses.tex}  % name of the review reponses file

% counters for reviewer points
\newcounter{reviewer}
\setcounter{reviewer}{0}
\newcounter{point}
\setcounter{point}{0}

\newcommand{\reviewersection}{\refstepcounter{reviewer}\setcounter{point}{0}
                  \section*{Reviewer \#\thereviewer:}}
% drawing from from http://tex.stackexchange.com/questions/2317/latex-style-or-macro-for-detailed-response-to-referee-report
\newcommand{\point}[2]{\refstepcounter{point} \bigskip \hrule \medskip \noindent 
        \slshape {\fontseries{b}\selectfont (\thereviewer.\thepoint) #1} #2} 
\newcommand{\reply}{\normalfont \medskip \noindent \textbf{Reply}:\ }   

% use this command in the text where a change addressing a reviewer point has occurred
\newcommand{\revpoint}[2]{\linelabel{rr:rev#1:#2}}
% and this one to refer to such a location
\newcommand{\revref}{{\iflatexml{}\else{(p.\ \pageref{rr:rev\thereviewer:\thepoint}, l.\ \lineref{rr:rev\thereviewer:\thepoint})}\fi}}

% or, this one to refer to a named linelabel
\newcommand{\llname}[1]{{\iflatexml{}\else{(p.\ \pageref{#1}, l.\ \lineref{#1})}\fi}}

% put this where the reviews are to appear (at the end?)
\newcommand{\includereviews}{
    \ifreviewresponses
    \clearpage
    \setcounter{page}{1}
    \setcounter{section}{0}
    \setcounter{subsection}{0}
    \nolinenumbers
    \begin{center}
      {\LARGE \bf Response to Reviews}
    \end{center}
    \input{\responsefile}
    \fi
}

% Useful shortcuts
\newcommand{\rollover}{ \reply{The reviewer makes an excellent point that we have missed out entirely.  We have made all the changes suggested, down to the minutiae \revref.} }
\newcommand{\playdead}{ \reply{The reviewer makes an excellent point.  We have made an utterly trivial change {\revref} that we think deals entirely with the concern raised.} }


% from http://tex.stackexchange.com/questions/43648/why-doesnt-lineno-number-a-paragraph-when-it-is-followed-by-an-align-equation/55297#55297
\newcommand*\patchAmsMathEnvironmentForLineno[1]{%
  \expandafter\let\csname old#1\expandafter\endcsname\csname #1\endcsname
  \expandafter\let\csname oldend#1\expandafter\endcsname\csname end#1\endcsname
  \renewenvironment{#1}%
     {\linenomath\csname old#1\endcsname}%
     {\csname oldend#1\endcsname\endlinenomath}}% 
\newcommand*\patchBothAmsMathEnvironmentsForLineno[1]{%
  \patchAmsMathEnvironmentForLineno{#1}%
  \patchAmsMathEnvironmentForLineno{#1*}}%
\AtBeginDocument{%
\patchBothAmsMathEnvironmentsForLineno{equation}%
\patchBothAmsMathEnvironmentsForLineno{align}%
\patchBothAmsMathEnvironmentsForLineno{flalign}%
\patchBothAmsMathEnvironmentsForLineno{alignat}%
\patchBothAmsMathEnvironmentsForLineno{gather}%
\patchBothAmsMathEnvironmentsForLineno{multline}%
}

% 
% \begin{document}

\section*{From the editor}

\begin{minipage}[b]{2.5in}
  Resubmission Cover Letter \\
  {\it PLoS Genetics}
\end{minipage}
\hfill
\begin{minipage}[b]{2.5in}
    Department of Molecular and Computational Biology\\
    University of Southern California, \\
    \emph{and} Department of Evolution and Ecology\\
    University of California at Davis\\
  \today
\end{minipage}
 
\vskip 2em
 
\noindent
{\bf To the Editor(s) -- }
 
\vskip 1em

\begin{quote}
As can you see from the reviews, two general points were raised by all
reviewers and I feel that thoroughly addressing both of these points poses a
prerequisite for further consideration of the manuscript:

(i) There needs to be some sort of numerical verification (e.g. with
individual-based simulations) to test the main conclusions, check their
robustness, and identify the parameter ranges for which they hold.

(ii) The presentation of the main result needs to be improved. In particular,
all reviewers feel that the main result can be derived in a simpler and more
intuitive way. I agree with them that at present the manuscript is overly
technical and would benefit strongly from a more intuitive derivation of the
results and discussion of the assumptions under which they hold. The reviewers
provide detailed suggestions on how this can be achieved.
\end{quote}


Changes of note are:

\begin{itemize}

  \item We have done a large number of simulations, and compared the results to the theory;
    see figures~\ref{fig:sim_probs} and \ref{fig:sim_times}.


  \item Several reviewers pointed out that in approximating $\E[1-(1-p_e)^K] \approx \E[K] p_e$
    we were neglecting the case that $s_m \ll s_p$.
    This was pur laziness on our part,
    which we have rectified; see the new equation~\eqref{eqn:twocase_migrate}.


\end{itemize}



%%%%%%%%%%%%%%
\reviewersection

%%%%%%
\point{}{
    Am I right that the basic argument is the
    following:

    A migrant family in unfavorable territory will be typically of size $n \sim 1/s$
    ($s=s_m$), stick around for a time $\tau \sim 1/s$ and be spread out over $\ell \sim \sigma/\sqrt(s)$. 
    Given an equilibrium density $\rho*q(x)$, we need an arrival rate
    of migrant families of
    \[ k ~ \rho q(x) l/n/\tau \sim \rho q(x) l s^2 \]
}

\reply
The factor of $\ell$ shouldn't be there:
suppose that $X$ is a Poisson process of ``centers'' of rate 1, 
and that around each center of $X$ we distribute $n$ points uniformly within a circle of radius $\ell$;
the expected number of points in a set of area 1 is $n$, regardless of $\ell$.
The right thing to do with the factor of $\tau$ is harder to intuit:
it is clear that $\rho q(x)$ should be $E[K]$ multiplied by the number of families present;
and it might be argued that the number of families present should be the influx of families
multiplied by the time they stick around for,
but given that the typical size of a family decreases once it arrives,
the correct way forward becomes non-intuitive, to us, at least.

%%%%%%%%
\point{}{ 
    The probability of establishment of at least one member of family is then
    \[ P_{est} = 1 - \exp(-p n) \sim 1 - \exp(-p_e/s) \]
    where $p_e$ is the establishment probability of a single individual. The result
    for $\migrate$ will then depend on whether $p_e/s$ is large or small.
    For $p_e/s$ large, $\migrate = \rho q(x) s^2 \ell$,
    otherwise,
    $\migrate = \rho q(x) s \ell p_e$.

    The latter is the case discussed in the manuscript. The other is important too
    and should be discussed. 
}

\reply
As discussed above, we have incorporated this good point; see around equation~\eqref{eqn:twocase_migrate}.

%%%%%%%%
\point{}{ 
    I don't seem to be able to identify the equivalent of
    $\ell$ in the author's formulae, but I believe it should be there. After all, 
    $\rho q(x)$ is a density and needs to be multiplied by a length (or area in d=2) to
    give a meaningful rate.
}

\reply
    See the response to point 1.1 above.

%%%%%%%%
\point{}{ 
    To assess the accuracy of the various approximations made, the process should
    be simulated and the measured rates explicitly compared to analytical
    predictions.
}

\reply
  We've done this; see above.


%%%%%%%%
\point{long range migration}{All results depend exponentially on the distance, which
      is a consequence of the model only allowing for short range dispersal. Rare
    long range dispersal will alter all these rates dramatically, see for example
    recent work by Hallatschek and Fisher.
}

\reply
This was already discussed in the discussion, but we've added a note about it earlier on, citing Hallatschek \& Fisher.
\llname{rr:longtailed}

%%%%%%%%
\point{Fig 4:}{ Why not use natural units for the axis of these plots,
  $R\sqrt(s)/\sigma$, $\rho \mu A/s$.
  }
  
\reply
Good idea, but we've decided to cut this figure.


%%%%%%%%
\point{in the discussion on page 10 "probability of parallel adaptation.."}{
    it should be noted that many of the scenarios at high mutation influx are quite
    hypothetical. While the colonization of migrants might dominate over local
    adaptation, this argument assumes that the migrants are at equilibrium
    frequency. The transient to reach this equilibrium might be much longer than
    the time required for de novo local adaptation.
}






%%%%%%%%%%%%%%  Reviewer 2
\reviewersection

\begin{quote}
Briefly, I think that the derivation of the main results (12), (21), and (25) can
be made much shorter and easier to understand (for both empiricists and the-
orists), and that there need to be simulations that quantitatively confirm these
results. Overly-detailed comments are below.
\end{quote}

\point{}{
    I think that the main result (12) can be derived much more simply, and
    also extended. By writing that the probability that a family that arrives
    at the patch is successful is approximately $ 2s_p \E [K] /\xi^2$, the authors have
    implicitly assumed that this quantity is $<\ll 1$. But this implies that the
    fluctuations in the number of B individuals arriving at the patch can be
    ignored, since even a large fluctuation does not saturate the probability
    of success. This is valid for $s_m \gg s_p$. Given that the fluctuations can be
    ignored, (12) can be derived immediately from $q(y)$ -- simply apply Fick’s
    Law to $q(y)\rho$ to get that the flux of new B individuals is $\sigma^2 |q'(y)|\rho/2$, and
    then multiply by the success probability $2s_p/\xi^2$. (Actually, the result is
    slightly different from (12) -- see X below.)
}

\reply
This argument falls in the trap of treating each migrant individual as independent,
ignoring the correlations of arrival times induced by families;
see the discussion in which we now set this up as a straw man in XXX.

\point{}{
    However, I think we should also care about the opposite limit, $s_m \ll s p$. In
    this case, the fluctuations do matter -- arrivals typically occur in clumps
    (families) large enough so that success is almost certain conditioned on
    arrival. Since on average each family consists of $\E[K]$ individuals, the
    family flux is just the individual flux divided by $\E[K]$. $\E[K] = \xi^2 /(2s_m)$,
    so the result is the same as above, except with $s_p$ replaced by an additional
    factor of $s_m$. (This could be made more formal, but I don’t think that’s
    what the authors are going for.) In the intermediate case $s_m \approx s_p$ which
    might be the most relevant for the mice, the two limiting regimes cross
    over smoothly. So for the first limit ($s_m \gg s_p$), all you need is $q$, and
    for the second, you only need to add basic facts about branching random
    walks (ie, limiting number distribution and localization at long times).
}

\reply
In this argument, the reviewer is taking into account clumpiness for establishment of migrants,
but still not for the time interval
between arrivals of families.

%%%%%%%%
\point{}{ 
    As it stands, $\migrate(y)$ is the rate of successful migrants at an infinitesimal
    area at distance $y$, so it needs to be integrated over an area to get the rate
    for a whole patch. The authors don’t do this, and as a result, expressions
    like (14) are dimensionally inconsistent. It’s a little annoying to define the
    right area of integration, but it’s straightforward to argue that up to a
    constant the answer
     should be given by integrating $\migrate(y)$ over a region
     of thickness $\sigma/\sqrt{2sm}$ around the boundary of the patch. I think it’s easier
    to work with fluxes and just integrate over the boundary of the patch, with
    no worries about thickness, so that’s why I used them above. (Of course,
    you get the same answer either way, except for small or weirdly-shaped
    patches, where either or both can break down.)
}

%%%%%%%%
\point{}{ 
    I think that the authors need to show simulation results confirming the
    analytic approximations. Spatial simulations are a pain, but it’s important
    to have a (somewhat) independent check on the math and assumptions. At
    a minimum, I think that there needs to be a figure showing that they have
    the right expression for the rate of successful migrants (12) and migration
    sojourn times (17) over a range of parameter values. If it’s not too much
    work, I think it would be really nice to have a figure that doubles as a kind
    of graphical summary, showing a scattering of patches in two dimensions,
    with one starting as the homeland of the B allele and the rest colored
    according to their probability in simulations of being colonized by that
    original allele vs new mutations, and a curve marking the region predicted
    as being colonized by the analytical approximations, or something like
    that. But I understand if the authors don’t want to spend their time on
    that.
}

%%%%%%%%
\point{}{ 
    It would be nice to have a table of symbol definitions.
}

\reply
We've tried to say in words what each symbol is when we use it as much as possible,
so we hope this isn't necessary.

%%%%%%%%
\point{}{ 
    And speaking of symbols, I think it’s a bit nicer to write things in terms
    of the characteristic patch length $a \sim A^{1/d}$, rather than the area. This
    avoids having to talk about one-dimensional areas \ldots
}

\reply
Good point, but $A$ is what shows up naturally in $\mutrate$, so we've left it as-is.

%%%%%%%%
\point{}{ 
     It would be nice to have a clear, concise summary of the quantitative
     assumptions needed for the results (all the stuff like $a \gg \sigma/\sqrt{2s_m}$ or
     whatever).
 }

\reply
See text added in \nameref{ss:assumptions}.

%%%%%%%%
\point{"Length of the hitchhiking haplotype"}{
    I would focus more on the asymptotic expressions for $R \gg w  \gg \sigma/ \sqrt{2s_m}$, since that’s the regime
     where the analysis makes the most sense. Then the expressions become
     simpler and you can treat d = 2 as well. If you want, you could even
     lead with the intuitive explanation at the end of the section, and save the
     Bessel functions for an appendix.
}

 \reply

%%%%%%%%
\point{}{ 
     I think (24) follows directly from Barton, Heredity 1979. (If you look at
     the unnumbered equation above Fig. 1 in that paper and plug in $x \sim a$,
     you find the time scale $r_{eff}$.) I think that the displayed equation should just
     be $r_{eff} \approx r\sigma/\sqrt{s}/a$. (I feel like the extra correction terms and geometric
      factor go against the spirit of the paper, and in any case are probably
      small compared to the inaccuracies introduced just by applying the model
      to any real system.) I think you can cut essentially everything before
      the simple explanation just above (24); if people want the mathematical
      details, they can read Barton. If I am misunderstanding things and there
    is a new result here, rather than just a good new exposition of an existing
    result, there should be a figure showing simulations confirming it.
}

\reply

%%%%%%%%
\point{In “Applications”,}{the authors spend a lot of time introducing the mice
    and calculating the values for Fig. 6, but then they don’t show where the
    actual data lie on those plots. Is there enough information to do some
    kind of comparison? How close are the nearest samples to each other?
    Have no examples of alleles spreading by migration been found? If not,
    how much of a bound does this put on the parameter values? Is there any
    data on the genetic map length of the region of reduced diversity around
    the selected locus?
}

\reply

%%%%%%%%
\point{}{ 
    A few sections or passages seem out of place:
    \begin{itemize}
        \item[A.]  I think “Simulation methods” should go in the “Materials and meth-
            ods” section, or at least not right between “Establishment of a locally
            adaptive allele due to mutational influx” and “Establishment of a lo-
            cally adaptive allele due to mutational influx”.

        \item[B.] The sentence on p7 beginning “By our definition of fitness. . . ”.

        \item[C.]  The paragraph starting “Also note that. . . ” on p11.
    \end{itemize}
}

\reply

%%%%%%%%
\point{Fig. 4:}{$R/\sigma$ should have dimensions of $\sqrt{\text{time}}$. In the caption, I think it’s
    misleading to say that $\migrate$ is ``relatively insensitive to $s_m$''. It’s exponen-
    tially sensitive! I don’t understand the blue numbers in the right-hand
    panel. It might be easier to interpret these if they were contour plots
    of the total time to adaptation, colored according to the value of (14),
    although I’m not sure. Also, they aren’t really ``phase diagrams''.
}

\reply

%%%%%%%%
\point{Top of p11 (just above “Multiple patches”):}{should be $\sqrt{2s_m}/\sigma$.
}

\reply

%%%%%%%%
\point{Between (49) and (51):}{should be “Green’s”.
}

\reply

%%%%%%%%
\point{(51)-(55)}{are missing factors of $\sigma$.
}

\reply

%%%%%%%%
\point{}{ 
 In (56) and following, it would be clearer to use different symbols for scaled
and unscaled distances.

}

\reply

%%%%%%%%%%%%%%  Reviewer 3
\reviewersection


\point{1a.}{
        What I miss is a clearer discussion of the scope and limits of the
    main results. Take Eq. 13: in its revised form, this is a non-monotonic
    function of the selection coefficient $s_m$. I am almost sure that this is
    an artefact and that the parameter range of the theory needs to be
    restricted to the part where the transition probability declines with $s_m$.
    But it is not easy to see how this needs to be done in a proper way. We
    certainly need a minimal distance among patches. On top of that, we may
    need a lower bound for $s_m$. Both these restrictions (or any other) should
    not be just ad-hoc, but motivated from within the theory (see my remarks at eq. 13).
}

\reply
Very good points.  
We apologize for not catching the nonmonotonicity earlier.
We've added a section, \nameref{ss:assumptions}, discussing this point,
as well as a note regarding the nonmonotonicity wrt $s_m$
\llname{rr:sm_condition}.

\point{1b.}{
    Linked to this: Figure 4, right panel, shows results in a parameter
    range where (I think) the theory does not apply.
}

\point{1c.}{
    Also linked to this: I believe your theory is sound, but it is really
    tricky to clearly define its range of application and also to see how good
    it is quantitatively. Bottom line: we need individual-based simulations to
    test at least some of the main conclusions. E.g.: do simulations for 2
    patches in 1d to test the predicted demarcation line between the
    "mutation" and the "migration" regime. Do this for several parameter
    choices to test the limits of the theory.
}

\point{2.}{
    Section "outflux of migrant alleles": I found this section much less
    clear than others and have made suggestions for a clearer presentation
    (see ms).
}

\point{3.}{
    Section "length of shared haplotype": This is very handwavy and
    (without simulation backup) I find it hard to believe that these results
    are accurate. Since it is also not central to the paper, the section
    should maybe just be dropped.
}

%%%%%%
\point{[beginning comments in the pdf]}{
``Do you assume a Gaussian kernel? Or is this irrelevant? Which properties of the kernel are required? Only existence of a finite $\sigma^2$?''
}

\reply
We've tried to make it more clear here 
\llname{rr:longtailed}
exactly what we sweep under the rug later.

%%%%%%
\point{}{``This is not sufficient: the patch also needs to be large enough that $p_e$ is (almost) reached at its center. Generally, the approximation should over-estimate the true value.''}

\reply
Good point; we have adjusted this slightly. \llname{rr:p_estab}

%%%%%%
\point{[more complete information about simulation model]}{}

\reply
We've added more description of how we did the simulations
\llname{rr:sim_details},
and don't think the choice of a bounded kernel is particularly salient
(given that it's not long-tailed, of course).

%%%%%%
\point{[comments near equation \eqref{eqn:eqfreq}]}{}

\reply
Both comments are excellent. 
We have added discussion of the intuition and the restrictions on $|x|$, as suggested
\llname{rr:q_form}.

%%%%%%
\point{The outflux of migrant alleles}{
    \begin{quote}
        This section is less clear than the previous one. [...]
    I think you could devote the formal derivation to the "Methods" and argue on an intuitive level in the main text:

    First, you should clearly spell out what the approximation is. You use the trunk + outgrowth picture of the branching process. Then you assume that the mutant family will only be able to cross to the other patch if the trunk hits this new patch. And when the trunk hits, you assume that all the outgrowth branches also enter it. I.e.: Eventually, either the whole family is "in" and then stays in (as in Fig 3) or no part ever enters. This approximation should be fine if the size of the new patch is sufficiently large.

    As a consequence of this, we can just follow the trunk, which moves by simple Brownian motion with an exponential life span. This defines h(x,y) as probability that the trunk arrives at the new patch.

    Now, how can we relate $h(x,y)$ to $u(x,y)$? - As follows: Once the trunk reaches the new patch (and assuming $s_m$ selection everywhere) it will live there for an exponential time with mean $1/2s_m$. Since it "lives or dies" with its entire outgrowth family (of expected size $\E[K]$) it thus contributes to the density like $u(x,y) = h(x,y) \E[K] / (2s_m)$. This motivates Eq. 11. (no need to introduce $g(A)$ in the main text)
    \end{quote}
}

\reply
   Good suggestion; as discussed above,
   we've taken what we think is a similar approach, 
   providing a more intuitive arguemnt (in \nameref{ss:heuristics})
   and merged the section referred to here with the Methods \llname{rr:moved_argument}.
   
%%%%%%%%
\point{[awkward sentence]}{}

\reply
Fixed? \llname{rr:awkward}

%%%%%%%%
\point{[expected total number]}{if y is a point, this number would be 0 (its a density)}

\reply
We have added the word ``informally''. \llname{rr:density}


%%%%%%%%
\point{[equation \eqref{eqn:migrate}]}{ 
    \begin{quote}
    This function is non-monotonic in $s_m$ and I think this needs to be discussed: can it really be that $\migrate$ increases with $s_m$ or is this an artifact?

    In general, we need a clearer idea about the limits of this theory. There are several things to consider:
    \begin{itemize}
        \item equation (3) for the expected mutant density is only asymptotic. Therefore, we probably need to require $|y| \gg \sigma / \sqrt(2s_m)$.

        \item the approximation $g(A) \sim 1/2s_m$ in the appendix requires that the patch size is much larger than $1/s_m$

        \item in your arguments, you use the outflow of families from a ball with (large) radius $r_0$ [although I do not see how this figures in]

        \item over short ranges, results must depend on details of the dispersal kernel anyway. [Depending on the shape of the distribution - heavy tail - this even holds for large distances]. As mentioned above: spell out assumptions on the kernel.
    \end{itemize}

    So: what is the minimal distance $|y|$ for the result to be valid? And: does this minimal distance eliminate the anomaly of an increasing $\migrate$ with $s_m$ for small $s_m?$ Alternatively: is there also a lower limit for $s_m$? I have noted that you do assume above that $s_m$ is large enough to avoid local fixation - but I'm not sure that this comes down to the same thing. 
\end{quote}
}

\reply



%%%%%%%%
\point{[expression for the outflux of migrants]}{do we need this for anything?}

\reply
We don't need it for anything else, 
but we think it adds sufficiently to the intuition about the process
that we'd like to keep it in.
In practice it could provide a good reality check.
\plr{alternatively, compute this for the mice example?}

%%%%%%%%
\point{[figure 1 showing unimportance of square root term]}{}

\reply
Clarified. \llname{rr:sqrt_term}

%%%%%%%%
\point{``the assumption of well-separated patches''}{ 
}

\reply
We have changed this. \llname{rr:separated_patches}


%%%%%%%%
\point{[figure 4, the contour plots]}{
}

\reply
Good point; we didn't notice that after correcting $\migrate$ the parameter regime on the right didn't make sense.
We have substituted XXX for these.

%%%%%%%%
\point{[\eqref{eqn:total_migrate} assumes no interference}{ 
}

\reply
Good point, but one we don't think we need to get into; we've added a small caveat.
\llname{rr:interference}


%%%%%%%%
\point{}{ 
    One message here seems to be that "migration vs mutation" is independent of the density. As far as the model goes, this is correct. However, in Nature selection will typically be density dependent, usually with stronger deleterious selection (higher $s_m$) at higher densities. One could therefore suspect that higher densities will often favor mutation relative to migration.
}

\reply
Interesting point,
but we think its applicability depends on what densities are being compared.
If it is in two different species, then $s_m$ must clearly be measured separately in the two,
so this is not a worry.
If it is two populations of the same species,
then one might also argue that lower densities implies worse habitat,
and so stronger deleterious selection.
We don't see anything straightforward we can say here, 
other than adding this caveat later. \llname{rr:popdensity}

%%%%%% p.12
\point{[justify $\E[K]\approx 1/s_m$]}{}

\reply
See \eqref{eqn:mean_K}, and modified discussion around \llname{rr:EK}.

%%%%%%%%
\point{}{ 
    There is a lot of handwaving in this section. Although each single step has some plausability, the final result seems to depend on a lot of assumptions. There is also an important difference to the other sections. Most of the other results are valid for large distances, where we can safely use asymptotic forms of the distribution (which arguably are quite universal). In contrast, this section relies on short-range dispersal (limited excursions of B-mutant lines to the surrounding area, where b is common). This is far less universal and should depend on details of dispersal.

    Also, it seems to me that another route to break down LD is by b-immigrants to the patch, who will life there for an exponential life span and can recombine before they die out. Note that, in the pocket mouse example below, the frequency of dark mice on the lava flows is only 86\%. This should result in considerable recombination "on site".

    So, in summary: I am sceptical concerning the results of this section. They are preliminary at best and need further study. I therefore suggest to cut them from the paper (or reduce it to some brief remarks).
}

\reply

%%%%%%%%
\point{Dominance.}{ 
    the deleterious effect could be recessive and the beneficial one still (at least partially) domninant.
}

\reply
Added that note. \llname{rr:dominance}



% \end{document}
