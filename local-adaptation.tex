\documentclass{article}
\usepackage{amsmath,amssymb}

\newcommand{\var}{\mathop{\mbox{Var}}}
\renewcommand{\P}{\mathbb{P}}
\newcommand{\E}{\mathbb{E}}
\newcommand{\R}{\mathbb{R}}
\newcommand{\N}{\mathbb{N}}
\newcommand{\one}{\mathbf{1}}


\begin{document}

Suppose that mutants have selective (dis-)advantage $s(x)$ at location $x$,
that the symmetric dispersal kernel is $\kappa(\cdot)$, and that the mean-squared dispersal distance is $\sum_y y^2 \kappa(y) = \sigma^2$.
Then in the large population limit, if $\phi(t,x)$ is the proportion mutants at location $x$ at time $t$,
if $s$ and $\sigma^2$ are of the same (small) order,
(see Peter's slides from Berkeley probability seminar)
\begin{align}
    \partial_t \phi(t,x) &\approx \sum_y s(x) \kappa(y-x) (1-\one_{x=y}) \phi(t,y)(1-\phi(t,x)) 
            + \sum_y \kappa(y-x) (1-\frac{1}{N}\one_{x=y}) (\phi(t,x)-\phi(t,y))  \\
        &\approx s(x) \phi(t,x) (1-\phi(t,x)) + \frac{1}{2} \sigma^2 \partial_x^2 \phi(t,x) .
\end{align}

\end{document}
