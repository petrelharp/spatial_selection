\documentclass{article}
\usepackage{amsmath}
\usepackage{graphicx}
\begin{document}

As I understood it, Dmitri's intuition (deriving from simulations) says that
in real populations, the plausibility of parallel adaptation 
is determined by the size of the mutational influx $2N\mu$,
and that additional opportunity for parallel adaptation caused by spatial structure is relatively unimportant,
at least as far as the plausibility (not necessarily the degree) goes.
This might be because: if $2N\mu \gg 1$,
then there are many successful new mutants {\em every generation},
and so even in the fairly brief time it takes a single mutation to fix locally,
new mutations will arise and fix locally.
In this case, we get parallel adaptation without invoking spatial structure.
Conversely, if $2N\mu \ll 1$,
then spatial spread allows more time for new mutants, 
but since fixation is still fast, the additional amount of time is still insignificant.

Clearly, there are parameter ranges where the above holds, and does not hold,
but the question is how restrictive are those conditions.
I am turning this into three questions:
\begin{enumerate}
  \item When will other new mutations arise in the amount of time it takes, say, the first mutation to fix, locally?
  \item When does spatial structure significantly increase the probability of parallel adaptation?  
  \item When does spatial structure significantly increase the degree of parallel adaptation? 
    (i.e.\ number of origins)
\end{enumerate}
Throughout I'll work in two dimensions.
Let $\rho$ be the population density, 
$\sigma$ be the dispersal distance, 
$\mu$ be the mutation rate,
$s$ be the selection coefficient,
$N$ be the total population size,
$A=N/\rho$ be the total area occupied,
and $r = \sqrt{A}/\sigma$.
I assume that $r$ is not small, since otherwise the problem is not spatial.

The first question is something we did not include explicitly in our first paper.
The area occupied by an advantageous mutation grows quadratically, 
but only after a (random) initial period of local fixation.
We can call ``local fixation'' 
the time when the growth of a panmictic and a spatially restricted family
begins to differ significantly,
which occurs when the number of individuals is $\sigma^2 \rho$.
This time is thus approximately
\[
  T_l = \frac{1}{s} \log (\sigma^2 \rho) .
\]
(The standard deviation of this time is of the same order; let's ignore the randomness.)
On the other hand, the time we have to wait to see a new, successful mutation is on the order of
\[
  T_n = \frac{1}{4N\mu s} .
\]
Therefore, we will see new mutations in the time it takes a new mutation to fix locally
(and hence, the time it takes a wave to get going)
if
\[
    T_l > T_n ,
\]
or since $\rho = N/A$,
\begin{equation} \label{eqn:tlggtn}
  2 \log(\sigma / \sqrt{A}) + \log(N) > \frac{1}{4\mu N} .
\end{equation}
A reasonable range for $r=\sqrt{A}/\sigma$ is $10 \le r \le 1000$,
over which range $4 < 2\log(r) < 14$, while perhaps $10 \le \log(N) \le 20$.
In other words, to get new mutations in the time it takes to locally fix,
we'd need $1/(4 N (\log N + 2 \log r)) < \mu$.
Setting $\mu=10^{-8}$ and taking $r$ in this range
tells us that we only get multiple single base mutations in this time 
if $N$ is larger than about a million.


For the second and third questions,
we need to compare the panmictic case to the spatial case.
In the panmictic case, the offspring of a family continues to grow logistically,
so the amount of time it takes to go from frequency $\sigma^2 \rho / N$ to frequency $1/2$ 
is approximately
\[
  T_p = \frac{1}{s} \log (N/\sigma^2 \rho) = \frac{1}{s} \log(A/\sigma^2) .
\]
On the other hand, the spatially restricted family will grow at speed $v=\sigma \sqrt{s}$, 
so the numbers grow like
\[
  \rho \sigma^2 s \left(t - s^{-1/2} \right)^2,
\]
with time measured in units after local fixation.
Therefore, the additional time it takes to reach frequency $1/2$ is 
\[
T_s = \sqrt{\frac{N}{2\sigma^2 s \rho}} + \sqrt{\frac{\rho}{s}} .
\]
This will give us significant extra time if $T_s \gg T_p$, i.e.\ if
\begin{equation} \label{eqn:tsggtp}
  % \sqrt{s}\left( 1 + \sqrt{\frac{A}{2\sigma^2}} \right) \gg \log(A/\sigma^2) . %exact
  \sqrt{\frac{s}{2}} \gg \frac{ \log(A/\sigma^2) }{ \sqrt{A/\sigma^2} }. %approx
\end{equation}
This extra time will be enough for some mutations if $T_s > T_n$, or
\begin{equation}
  \sqrt{\frac{sA}{2\sigma^2}} > \frac{1}{4N\mu} .
\end{equation}

Looking again at equation \eqref{eqn:tsggtp},
the spatial restriction will give much more time if
\[
  s \gg 8 \left( \frac{\log(r)}{r} \right)^2,
\]
where $r=\sqrt{A/\sigma^2}$ is the radius of the species range in units of dispersal distance.
Here is a plot of $s = 8 \left( \log(r)/r \right)^2$,
for $10 \le r \le 1000$, giving the lower limit for possible values of $s$.
At smaller species ranges, this is unlikely to happen, 
but for species ranges hundreds of times the dispersal distance, it starts to become reasonable.
Humans and Drosophila are on the edge -- if $\sigma=10$km,
then the spatial distribution is giving extra time; 
but if $\sigma=100$km, perhaps it is not.

 \begin{figure}[hht]
  \includegraphics[width=0.8\textwidth]{min-s-versus-r.pdf}
 \caption{Plot of $s= 8 \left( \log(r)/r \right)^2$ against $r=\sqrt{A/\sigma^2}$.} 
 \end{figure}

To answer the second question, 
to expect parallel adaptation in the spatial case but not in the panmictic case,
we would clearly need $T_l \ll T_n$, so no new mutations are arising in the time to local fixation,
since this part of the process is shared by both.
Recall that this effectively puts an upper bound on $N$.
We'd also need
$T_s > T_n \gg T_p$,
or
\begin{equation}
  \sqrt{\frac{sA}{2\sigma^2}} > \frac{1}{4N\mu} \gg \log(A/\sigma^2) ,
\end{equation}
or
\begin{equation}
  % \sqrt{s/2} > \frac{1}{4 N \mu r} \gg 2 \frac{ \log(r) }{ r } .
  s > \frac{1}{8 N^2 \mu^2 r^2} \gg 8 \left(\frac{ \log(r) }{ r }\right)^2 .
\end{equation}
This puts an upper and lower bound on the mutational influx $4N\mu$:
taking the middle values in the figure, 
spatial structure significantly increases the chance of parallel mutations if $1/20 < 4N\mu < 800$.

For the third question,
to get an idea we can use the results on standing variation,
and treat the mutations arising while the first is fixing, if such exist, as ``standing variation''.
This means setting 
\[
  \lambda_0 = 4 \rho \mu s T_l = 4 \rho \mu \log(\sigma^2 \rho) .
\]
Recall the characteristic length solves $\lambda_0 \pi \chi^2 + \lambda \pi \chi^3/v = 1$.
Set $\chi_0 = 1/\sqrt{\pi \lambda_0}$, 
the characteristic length only from the mutations arising while the first fixes locally;
we are interested in the numbers of mutations arising relative to these,
which is about $x=(\chi/\chi_0)^2$.
This proportion is between zero and one and satisfies
$x = 1-\gamma x^{3/2}$, 
where 
\[
  \gamma = \frac{ \sqrt{s} }{ \sigma \sqrt{4 \pi \rho \mu} \log^{3/2}(\sigma^2 \rho) } .
\]
The relationship between $\gamma$ and $f$ is shown in this figure:

\includegraphics[width=1.0\textwidth]{f-versus-gamma.pdf}

showing that if $\gamma>1$ then $f$ drops off quickly.
If we then fix $\mu=10^{-8}$ and $\rho=1$ (working in units of $1/\sqrt{\rho}$), 
then we get the following picture of $\log \gamma$:

\includegraphics[width=0.8\textwidth]{gamma-contour-1e-8.pdf}

showing that $\gamma$ is mostly affected by $\sigma$,
needing something like $\sigma<20/\sqrt{\rho}$.
If $\mu=10^{-6}$ this drops to something like $2/\sqrt{\rho}$.

\paragraph{In summary,}
the conclusions are: (recall $r = \sqrt{A}/\sigma$)
\begin{enumerate}

  \item Multiple successful mutations in the time it takes a single mutation to fix locally
    are likely if $1/(4 N \log N) < \mu$; if $\mu=10^{-8}$ this means in populations larger than a million.

  \item Spatial structure can make parallel adaptation likely where it was unlikely under panmixia
    if the above does not happen and if
    $s > \frac{1}{8 (N \mu r)^2} \gg 8 \left(\frac{ \log(r) }{ r }\right)^2$;
    this latter restricts the range of mutational influxes, but not dramatically.

  \item Spatial structure can dramatically increase the number of parallel adaptations if $\sigma$ is small enough;
    the requirement is that $\sigma<C/\sqrt{\rho}$, for some $C$ around ten.

\end{enumerate}

It is easy to think of populations that satisfy any of these requirements;
some natural examples that come to mind seem to lie in the ambiguous region,
although others (e.g.\ {\it Drosophila}?) lie firmly on one side of the boundary or the other.

\end{document}
