\documentclass{article}

\usepackage{amsmath,amssymb}

\newcommand{\var}{\mathop{\mbox{Var}}}
\renewcommand{\P}{\mathbb{P}}
\newcommand{\E}{\mathbb{E}}
\newcommand{\R}{\mathbb{R}}
\newcommand{\N}{\mathbb{N}}
\newcommand{\one}{\mathbf{1}}


\begin{document}
\section{General relationship between various quantities}

We know that there are $\lambda_0$ successful standing variants per unit area;
suppose that there are $\nu_+$ successful new mutants per unit area.
Therefore, $\lambda_0+\nu_+$ is the mean number of successful mutants per unit area (duh),
and $1/(\lambda_0+\nu_+)$ is the mean area associated with a randomly chosen patch.
Let $p_0$ be the proportion of space occupied by standing variants, 
let $a_0$ be the mean area associated with a randomly chosen standing variant,
and let $a_+$ be the mean area associated with a randomly chosen new variant,
so that
\[
    p_0 = \frac{ \lambda_0 a_0 }{ \lambda_0 a_0 + \nu_+ a_+ } = \lambda_0 a_0 ,
\]
since the denominator is equal to 1.

We know how to calculate $p_0$, above, from thinking about the competing exponentials,
which allows us to find $a_0$.

Now, as in the first paper, the mean density of new patches is easy to calculate -- 
$\lambda \nu_+ = \E[\tau]$, where $\tau$ is the mean adaptation time of a point;
this is because 
\begin{align*}
    \P\{ \tau > t \} &= \P\{ \mbox{no points in $t$-cone} \} \\
            &= (1/\lambda) \P\{ \mbox{new successful adaptation arises in non-occupied $(t+dt,x+dx)$} \} ;
\end{align*}
integrating over $t$ produces the result.

Finally, the proportion of patches that come from new mutation is $\nu_+ / (\nu_+ + \lambda_0)$,
and we know that, conditioning on the type of a randomly chosen patch,
\[
    \frac{1}{\lambda_0+\nu_+} = \frac{\lambda_0}{\lambda_0+\nu_+} a_0 + \frac{\nu_+}{\lambda_0+\nu_+} a_+ .
\]
So, given $\lambda_0$, $\nu_+$, and $a_0$, we can find $a_+$.


\end{document}
