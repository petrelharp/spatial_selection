\documentclass{article}

\usepackage{amsmath,amssymb}

\newcommand{\var}{\mathop{\mbox{Var}}}
\renewcommand{\P}{\mathbb{P}}
\newcommand{\E}{\mathbb{E}}
\newcommand{\R}{\mathbb{R}}
\newcommand{\N}{\mathbb{N}}
\newcommand{\one}{\mathbf{1}}


\section{Notes on branching process approximation}

The branching process approximation will be good as long as the probability of "interactions" are small,
i.e.\ we can couple the branching process with the actual density--regulated model.
Let $Z_t(x)$ be the numbers of selected individuals in deme $x$.
If we assume that $Z_t$ is smooth enough, 
the numbers of nearby individuals to interact with is approximately $Z_t(x) \sigma^d$,
and the total number of individuals is about $\rho \sigma^d$,
so the rate of interactions is about $(Z_t(x) \sigma^d)^2/(\rho \sigma^d)$.
Now if $Z$ is a spatial branching process beginning from a single individual at $x=0$ at $t=0$, 
and growing at rate $r$, then 
\[ 
    Z_t(x) \approx \exp(rt - \|x\|^2/(2\sigma^2 t))/ (2 \pi \sigma^2)^{d/2} 
        \le e^{rt} / ( \sigma^2 t)^{d/2}
\]
and so we need $t$ such that
\[
    e^{2rt} t^{-d} << \rho \sigma^{3d} .
\]
or equivalently,
\[
    e^{2rt/d} t^{-1} << \rho^{1/d} \sigma^3 .
\]
This will be true if $t$ is of smaller order than the logarithm of the effective population density, $\log \rho \sigma^d$.

\subsection{On the random delay}

Let $Y_t$ be the total number of offspring coming from a single individual
(in the spatial, density-regulated model).
Then we know that $(Y_t/t)^{1/d} \to v$, 
but we saw from simulations that it looked like $Y_t^{1/d} \approx v(t-\tau)$,
where $\tau$ is the (random) establishment time.
(It would be useful to look at the coefficient of variation of $\tau$.)
Well, we know that at first, $Y_t \approx W e^{rt}$, 
where $W= \lim e^{-rt} Z_t$, and $Z_t$ is the associated branching process,
and $r$ is the growth rate (er, $r=s$?).
And eventually, $\partial_t Y_t^{1/d} \approx v$.
If this switchover happened at a fixed size $C$,
the size at which wavelike behavior took over,
i.e.\ when $W e^{rt} = C$,
then for $\tau = \frac{1}{r}\log(C/W)$ we'd have 
\[
    Y_t = \begin{cases}
        W e^{rt} \quad & t \le \tau \\
        (v (t-\tau))^d + C \quad & t > \tau ,
\end{cases}
\]
which if $C$ is small relative to the scale we look on will look as described above.
So the random delay is $(1/r)\log(C/W)$; and so the mean delay is something like $(1/r)\log C$
and the randomness is determined by $(1/r)\log (1/W)$.

What about how much variance is left in the branching process?
Well, since $Z_{t+s} = \sum_{k=1}^{Z_t} Z^k_{s}$, where $Z^k_s$ are iid copies of $Z_s$,
we have that $\var[Z_{t+s}|Z_t] = Z_t \var[Z_s]$,
and so the coefficient of variation of $Z_{t+s}$ given $Z_t$ is
\begin{align}
   \mathop{CV}[Z_{t+s}|Z_t] &= \frac{\sqrt{Z_t \var[Z_s]}}{ Z_t \E[Z_s] } \\
        &= e^{-rs} \sqrt{ \var[Z_s] } Z^{-1/2} \\
        &\simeq e^{-rt/2} \sqrt{ \var[W] } .
\end{align}

\end{document}
