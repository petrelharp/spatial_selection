\documentclass{article}
\usepackage{amsmath,amssymb}

\newcommand{\var}{\mathop{\mbox{Var}}}
\renewcommand{\P}{\mathbb{P}}
\newcommand{\E}{\mathbb{E}}
\newcommand{\R}{\mathbb{R}}
\newcommand{\N}{\mathbb{N}}
\newcommand{\one}{\mathbf{1}}

\newcommand{\sn}{\mathop{\mbox{sn}}}


\begin{document}

Suppose that mutants have selective (dis-)advantage $s(x)$ at location $x$,
that the symmetric dispersal kernel is $\kappa(\cdot)$, and that the mean-squared dispersal distance is $\sum_y y^2 \kappa(y) = \sigma^2$.
Then in the large population limit, if $\phi(t,x)$ is the proportion mutants at location $x$ at time $t$,
if $s$ and $\sigma^2$ are of the same (small) order,
(see Peter's slides from Berkeley probability seminar)
\begin{align}
    \partial_t \phi(t,x) &\approx \sum_y s(x) \kappa(y-x) (1-\one_{x=y}) \phi(t,y)(1-\phi(t,x)) 
            + \sum_y \kappa(y-x) (1-\frac{1}{N}\one_{x=y}) (\phi(t,x)-\phi(t,y))  \\
        &\approx s(x) \phi(t,x) (1-\phi(t,x)) + \frac{1}{2} \sigma^2 \partial_x^2 \phi(t,x) .
\end{align}

Note that if $\tau$ is the first time that the new patch $A$ is occupied,
and $Z_t$ is the state of the proces, then
\[
    \partial_t \P\{ \tau > t \} = \int \int_A s(x) \kappa(y-x) \E[ Z_t(dx) | \tau > t ] dy \P\{ \tau > t \}
\]
... so we want to find the first integral (and show that for $t>\epsilon$ it is approximately constant).
Since we are also taking the small-dispersal limit, this seems tricky.

Another option: we are trying to find the first time that a mutant appears in patch $A$ {\em and fixes locally};
if the benefit is small then there may be many unsuccessful migrants first.
Suppose we modify the process so that offspring appearing in $A$ have distribution {\em conditioned on local extinction};
then we suppose that each such individual has probability ${}\approx 2s_b$ of fixation,
so that the mean number of such offspring per generation times $2s_b$ gives us the rate we are looking for.
A supercritical branching process conditioned on extinction is a subcritical branching process,
with mean offspring number $\partial_z \E[z^N]\vert_{z=p}$, where $N$ is the offspring number and $p$ is the probability of extinction.
In the case of a binomial (check this?) branching, this changes $s$ to $-s$.

\section{On solving FKPP with varying selection}

Let $p(t,x)$ be the proportion of the allele under spatially varying selection $s(x)$ in one dimension,
so that (setting $\sigma^2=2$)
\[
    \partial_t p(t,x) = \partial_x^2 p(t,x) + s(x) p(t,x) (1-p(t,x)) .
\]
The stable distribution $\phi(x) = \lim_{t\to\infty} p(t,x)$ then solves
\begin{align} \label{eqn:definingphi}
    \partial_x^2 \phi(x) = - s(x) \phi(x) (1-\phi(x)) ,
\end{align}
with appropriate boundary conditions.
If we now assume that $s(x)$ is piecewise constant,
$s(x) = s_i$ for $x \in [x_i,x_{i+1})$, with $x_0=-\infty$ and $x_{n+2}=\infty$,
then the equation is integrable: if we multiply through by $\partial_x \phi(x)$ and integrate, then we get that
\begin{align} \label{eqn:conservation}
    ( \partial_x \phi(x) )^2  &= - \int^{x} s(x) \phi(x) (1-\phi(x)) \partial_x \phi(x) dx \\
        &= - \frac{1}{2} s_i \phi(x)^2 \left( 1 - \frac{1}{3} \phi(x) \right) - K_i \quad \mbox{for } x \in [x_i,x_{i+1}) \\
        &:= - V_i(\phi(x)) \quad \mbox{for } x \in [x_i,x_{i+1}) ,
\end{align}
where $K_i$ is a constant of integration and we define
\[
        V_i(\phi) = \frac{1}{2} s_i \phi^2 \left( 1 - \frac{1}{3} \phi \right) + K_i .
\]
Note that $V_i'(0)=V_i'(1)=0$, that $V_i(0)=K_i$ and $V_i(1) = K_i-s_i/6$.  
We will always have that $V(\phi) \le 0$.
(We have then that $\phi'' = - \partial_\phi V(\phi)$, the equation of motion of a particle in potential $V$.
Also note that this implies ``conservation of energy'', i.e.\ $( \partial_x \phi(x) )^2 + V(\phi(x))$ is constant.)
Rearranging, we get that $dx = d\phi / \sqrt{-V(\phi)}$, so where $\phi(x)$ is monotone, the inverse is
\[
    x(\phi) = x(\phi_0) \pm \int_{\phi_0}^\phi \frac{ d\psi }{ \sqrt{ -V(\psi) } } .
\]
For each $i$ then define the elliptic function
\[
    F_i(\phi) = \int_{\phi_i^*}^\phi \frac{ d\psi }{ \sqrt{ -V(\psi) } } ,
\]
where take the positive branch of the square root, and $\phi_i^*$ will be chosen later.
Then we have that
$x(\phi) - x(\phi_0) = \pm( F(\phi) - F(\phi_0))$,
or for an appropriate $x_0$,
\[
    \phi(x) = F^{-1}\left( F(\phi(x_0)) \pm (x - x_0) \right).
\]

We clearly want $\lim_{x \to \infty} \partial_x \phi(\pm x) = 0$, 
and $\lim_{x \to \infty} \phi(\pm x)$ to be zero or one depending on the sign of $s_0$ and $s_{n+1}$.
Since $(\partial_x \phi(x))^2 = -V(x)$, this implies that if $s_0<0$, then $K_0 = 0$,
while if $s_0>0$ then $K_0 = s_0/6$; and likewise for $K_{n+1}$.

We also require that $\phi(x)$ and $\phi'(x)$ are continuous.
Continuity of $\phi'(x)$ is equivalent to $V_i(x_{i+1}) = V_{i+1}(x_{i+1})$,
which we can rearrange to find an equation for $K_{i+1}$ in terms of $K_i$ and $\phi(x_{i+1})$:
\[
    K_{i+1} - K_i = (s_i - s_{i+1}) \frac{1}{2} \phi(x_{i+1})^2 (1-\phi(x_{i+1})/3) .
\]
What about $\phi(x_i)$?  Well, if $\phi(x)$ is monotone on $[x_i,x_{i+1})$ then we can wolog take $\phi_i^*=0$ or $1$ depending on the sign of $s_i$.
Otherwise, let $\phi_i^*$ be the (unique) root of $V_i$ in $[x_i,x_{i+1})$, so $V_i(\phi_i^*)=0$.
In this case, $\phi_i^*$ is the maximum or minimum of $\phi$ in the interval:
if $s>0$, then $\phi_i = \max\{ \phi(x) : x \in [x_i,x_{i+1})\}$. 
Recall we defined $F_i$ using $\phi_i^*$; now using the fact that $\phi$ is monotone with the opposite sign on either side of $\phi_i^*$,
$x_{i+1} - x_i = F(\phi(x_{i+1})) - F(\phi_i^*) + F(\phi(x_{i})) - F(\phi_i^*)$,
and that $F(\phi_i^*) = 0$,
we know that the length of the $i$th stretch is
\[
    x_{i+1} - x_i = \pm \left( F_i(\phi(x_i)) + F_i(\phi(x_{i+1})) \right).
\]

Note that all the $\pm{}$'s are easily relatable to the signs of $s_i$.
If we knew $\phi(x_1)$ and $\phi'(x_1)$, then we'd be all set.
Unfortunately, we are given $\phi(-\infty)$ and $\phi(\infty)$, and have to work inwards from the ends.
Monty deals with this by using symmetry, so he knows $\phi(0)$ and $\phi'(0)$.

We are interested in the following case:
\[
    s(x) = \begin{cases}
        +s_b \qquad \mbox{for } |x|<\epsilon \\
        -s_d \qquad \mbox{for } x<-\epsilon \; \mbox{or}\; \epsilon \le x \le R \\
        -s_* \qquad \mbox{for } R < x < R+2\epsilon \\
        -s_d \qquad \mbox{for } x>R+2\epsilon .
    \end{cases}
\]
We can solve the case with $s_* = s_d$, since $\phi'(0) = 0$ (as can Monty, although he doesn't say what it is).
We should be able to argue that if $R$ is large then the value of $s_*$ won't make much difference.

\subsection{Doing the integrals}

This from ``NEQwiki, the nonlinear equations encyclopedia'', \url{http://www.primat.mephi.ru/wiki/ow.asp?Korteweg-de_Vries_equation}.
We want to integrate
\[
    \int \frac{ d\phi }{ \sqrt{-V(\phi)} } = 
        \sqrt{2}{s} \int \frac{ d\phi }{ \sqrt{ \phi^2 (1-\phi/3) + C } } ,
\]
with $C=2K/s$.
Let $\phi^2(1-\phi/3)+C = (\alpha-\phi)(\phi-\beta)(\phi-\gamma)$,
and change variables first to $y^2=(\alpha-\phi$, 
and then to $x = y/\sqrt{\alpha-\beta}$, so that
\begin{align*}
    \frac{ d\phi }{ \sqrt{ \phi^2 (1-\phi/3) + C } } 
        = \frac{ - 2 dy }{ \sqrt{ (\alpha-\beta-y^2) (\alpha-\gamma-p^2) } } \\
        = \frac{ - 2 dx }{ \sqrt{\alpha-\gamma} \sqrt{ (1-x^2) (1-S^2 x^2) } } ,
\end{align*}
with $S^2 = (\alpha-\beta)/(\alpha-\gamma)$.
Now Jacobi's ``incomplete elliptic integral of the first kind'' is
\[
    F(x;k) = \int_0^x \frac{dt}{\sqrt{ (1-t^2)(1-k^2t^2) }} ,
\]
and the Jacobian elliptic function $\sn(x;k)$ is the inverse: $F(\sn(x;k);k) = x$.

As $k \to 0$, $\sn(x;k) \to \sin(x)$, while as $k \to 1$, $\sn(x;k) \to \sinh(x)$.


\end{document}
