\documentclass{article} \usepackage[landscape]{geometry} \begin{document}
% latex table generated in R 2.13.0 by xtable 1.5-6 package
% Thu Jun 30 06:13:14 2011

\input{helianthus-ex-table.tex}

\paragraph{Parallel adaptation between patches?}
All values are in kilometers (squared km, etc).
The parameters: 
mu is the migration rate (per generation),
rho is the population density,
sb is the selection coefficient in the patches of dunes (chosen so the probability of fixation is $2\times sb$),
sm is the selection coefficient inbetween,
R is the distance between the patches,
and A is the area around a patch that might contribute adaptive mutations.
{\tt clScale} is the length over which the cline between the two types is expected to drop off;
{\tt minPatch} is the minimum area predicted to be able to locally adapt;
{\tt mutIn} is the mutational flux into an as-yet-colonized patch;
{\tt migIn} is the migrational flux from one patch to another.

\input{helianthus-standing-ex-table.tex}

\paragraph{Parallel adaptation within the main range?}
Here {\tt chi} is the characteristic scale of parallel adaptation;
and {\tt Etau} is the mean time it would take to adapt.


\end{document}
