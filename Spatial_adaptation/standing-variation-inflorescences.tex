\documentclass{article}

\usepackage{amsmath,amssymb}

\newcommand{\var}{\mathop{\mbox{Var}}}
\renewcommand{\P}{\mathbb{P}}
\newcommand{\E}{\mathbb{E}}
\newcommand{\R}{\mathbb{R}}
\newcommand{\N}{\mathbb{N}}
\newcommand{\one}{\mathbf{1}}



\begin{document}

\section{Size of inflorescences}

Let the local rate of growth at location $x$ and time $t$ be given by $f(x,t)$,
and for each $x$ and $t$ let $(X^{(x,t)}_{s})_{s \ge t}$ have the distribution of a branching process begun with one individual at $x$ and time $t$,
and let $\kappa^{(x,t)}_s(dz) = \P\{ |X^{(x,t)}_s| \in dz \}$  be the marginal distribution of the size of $X^{(x,t)}$.
For a given mutation rate $\mu$ and population density $\rho(dx)$ we have a Poisson process of inflorescences of rate $\mu \rho$,
each of which having distribution $\kappa$.  
Then the set of inflorescences alive at time $t$ is Poisson, 
and the mean measure of such inflorescences is
\[ \nu(dx,A) = \mu \rho(dx) \int_{-\infty}^t \P\{ X^{(x,s)}_t \in A \} ds .\]
The first thing we'd want to do is to ignore the spatial extent of $X$ and only look at the total number of individuals,
so we need to know
\[ \int_{-\infty}^t \kappa^{(x,s)}_t(dz) ds . \]

Now assume everything is homogeneous, and let $Z_t$ be a continuous-time branching process begun at zero 
and having branching rate $1$ and offspring distribution $\{p_k : k\ge 0\}$.
Suppose that $Z$ has total lifetime $\tau$ and that $\tau$ is finite almost surely.
If we treat each inflorescence as a point mass, then the set of (location, mass) coordinates $(x,z)$
are Poisson with intensity $dx \otimes q(1,z) dz$ where $q(1,z)=q(z)$ is the Green function of $Z$, namely
\[  
  q(y,z) = \E^y \left[ \int_0^\tau \one(Z_t = z) dt \right] = \int_0^\infty P^t_{yz} dt .
\]
and $q$ solves $L q(y,z) = y \sum_k p_k (q(y+k-1,z)-q(y,z)) = \delta_{y=z}$.

If we are interested in the density of ultimately successful influorescences,
we need to futhermore weight by the probability that the offspring of any of the initial (assumed to be small) number of individuals in an inflorescence survive.
If $p_s$ is the probability of survival in the new selective regime,
since the probability that an influorescence of size $z$ survives in the new selective regime is $1 - (1-p_s)^z$,
the mean measure for successful influorescences is
\[
   \mu dx \int_0^\infty (1-F(1-p_s,t)) dt ,
\]
where $F$ is the generating function for $Z_t$ (in the old selective regime),
\[
  F(a,t) = \E[a^{Z_t} \; | \; Z_0 = 1] .
\]
Note that 
if $\phi$ be the generating function for the offspring distribution in the old selective regime,
then $F$ solves
\[
   \partial_t F(a,t) = \phi(F(a,t)) - F(a,t).
\]
If $\phi_+$ be the generating function for the offspring distribution in the new selective regime,
\[
  \phi_+(a) = \sum_k p_k a^k .
\]
Now, $p_s$ is the minimal solution to $\phi_+(a) = a$.


\end{document}
